% latex char_ver1
% dvipdfm char_ver1

\documentclass[12pt]{article}
\usepackage{amsmath,amssymb,amsthm}
\usepackage{rotating}
%\usepackage{graphicx}
\usepackage{float}
%\usepackage{subfigure}
\usepackage{epsfig,color,amsfonts}
\usepackage{caption}
\usepackage{array}
\usepackage{booktabs}
\usepackage{graphicx}
\usepackage{multirow}
\usepackage{accents}
\usepackage{extarrows}

\newcommand{\blue}[1]{\textcolor{blue}{#1}}
\newcommand{\red}[1]{\textcolor{red}{#1}}

\captionsetup{font={footnotesize}}
%\textheight=9in
%\textwidth=6.5in
%\topmargin=0pt
%oddsidemargin=0pt
%\evensidemargin=0pt
\def\refhg{\hangindent=10pt\hangafter=1}
\def\refmark{\par\vskip 2mm\noindent\refhg}
\def\frac#1#2{{#1\over #2}}

\newtheorem{lem}{Lemma}
\newtheorem{thm}{Theorem}
\newtheorem{coro}{Corollary}
\newtheorem{prop}{Proposition}
\newtheorem{rem}{Remark}
%\renewcommand{\arraystretch}{1.7}
\bibliographystyle{model1a-num-names}

\begin{document}


\noindent \\
{\large {\underline{The Manuscript ID: RJournal 2020-173:}}}\\\
		
		\noindent Type of manuscript: Article
		\begin{center}
			WLinfer: Statistical Inference for Weighted Lindley Distribution
		\end{center}
		Yu-Hyeong Jang, SungBum Kim, Hyun-Ju Jung, and Hyoung-Moon Kim \\ \\
{\large	\underline{Authors' Reply to Comments from the editor}} \\
\medskip


We appreciate the constructive comments and suggestions on our manuscript entitled "WLinfer: Statistical Inference for Weighted Lindley Distribution". According to the editor’s instructions, we have made the following revisions on this manuscript:\\


\begin{itemize}
  \item ``The bootstrap code examples take too long to run". 
  \\
  \\  -- Our previous bootstrapping code included an unnecessary for-loop statement, which caused the computational inefficiency. We have now greatly enhanced the speed of bootstrapping by replacing the for-loop part with \texttt{boot.ci()} function in the \textbf{boot} package.\\
  With this upgraded code, it only takes about 0.4 seconds to obtain a bootstrap confidence interval in the example when the bootstrap sample size is 10,000. This measurement was made on a computer with 64 bit Windows 10 system, 8 GB RAM and Intel i5-7200U CPU @ 2.71 GHz. You can find the attached zip file for the R package, WLinfer, which is submitted today at CRAN.
  
  \item ``Some prompts should be given to the user detailing the $\%$ completed".\\
  ``Computational efficiency should be addressed in the article".\\
  \\--  As far as we understand, these two comments require us to complement the slow speed of bootstrapping. We think that the drastic increase in speed shall remove the need to do that. 
  
  \item We modified the subsection 'Bootstrap confidence intervals' on page 4. \\
  \\
  --We mistakenly explained that the percentile boostrap CI was used, even though we actually employed the `basic' bootstrap CI. We did make a correction to it and refer readers to a book wrttien by A.C. Davison and D.V. Hinkley for more details on bootstrap CIs.
\end{itemize}



\end{document} 