%%%%%%%%%%%%%%%%%%%%%%%%%%%%%%%%%%%%%%%%%
% Thin Formal Letter
% LaTeX Template
% Version 2.0 (7/2/17)
%
% This template has been downloaded from:
% http://www.LaTeXTemplates.com
%
% Author:
% Vel (vel@LaTeXTemplates.com)
%
% Originally based on an example on WikiBooks 
% (http://en.wikibooks.org/wiki/LaTeX/Letters) but rewritten as of v2.0
%
% License:
% CC BY-NC-SA 3.0 (http://creativecommons.org/licenses/by-nc-sa/3.0/)
%
%%%%%%%%%%%%%%%%%%%%%%%%%%%%%%%%%%%%%%%%%

%----------------------------------------------------------------------------------------
%	DOCUMENT CONFIGURATIONS
%----------------------------------------------------------------------------------------

\documentclass[10pt]{letter} % 10pt font size default, 11pt and 12pt are also possible

\usepackage{geometry} % Required for adjusting page dimensions

%\longindentation=0pt % Un-commenting this line will push the closing "Sincerely," to the left of the page

\geometry{
	paper=letterpaper, % Change to letterpaper for US letter
	top=3cm, % Top margin
	bottom=1.5cm, % Bottom margin
	left=4.5cm, % Left margin
	right=4.5cm, % Right margin
	%showframe, % Uncomment to show how the type block is set on the page
}

\usepackage[T1]{fontenc} % Output font encoding for international characters
\usepackage[utf8]{inputenc} % Required for inputting international characters

\usepackage{stix} % Use the Stix font by default

\usepackage{microtype} % Improve justification


%----------------------------------------------------------------------------------------
%	YOUR NAME & ADDRESS SECTION
%----------------------------------------------------------------------------------------

\signature{Barbara Lerner} % Your name for the signature at the bottom

\address{Computer Science Department \\
Mount Holyoke College \\ 
South Hadley, MA  01075} % Your address and phone number

%----------------------------------------------------------------------------------------

\begin{document}

%----------------------------------------------------------------------------------------
%	ADDRESSEE SECTION
%----------------------------------------------------------------------------------------

\begin{letter}{Michael Kane \\ Editor-in-Chief \\ The R Journal} % Name/title of the addressee

%----------------------------------------------------------------------------------------
%	LETTER CONTENT SECTION
%----------------------------------------------------------------------------------------

\opening{Dear Professor Kane,}

% Authors need to make a strong case (in a motivating letter accompanying a submission) for such introductions, based for example on novelty in implementation and use of R, or the introduction of new data structures representing general architectures that invite re-use. 

There is a growing recognition of the need for scientific results to be trustworthy and reproducible.  A key element for both of these goals is knowing the provenance of the scripts and their results:  precisely how were the results computed.  On behalf of my co-authors, I am submitting "Making Provenance Work for You" as an article describing several related add-on packages for R.  These packages implement provenance collection for R scripts and tools that use the collected provenance to provide novel functionality for R programmers. 

More specifically, the paper describes rdtLite, a tool that collects provenance as an R script executes.  The provenance includes information about the script, the environment in which it is executed, the input and output of the script, and an execution trace of the top-level statements within the script.  The provSummarizeR package provides a concise textual summary of the collected provenance.  The provViz package provides a graphical visualization of the provenance.  The provDebugR package uses the collected provenance to help programmers debug their code.  The provExplainR package compares the provenance from two executions to help the programmer understand changes between two executions.  The paper also describes two packages that can help others develop new provenance-based tools: provParseR provides a convenient API to the provenance collected by rdtLite, and provGraphR provides an API to perform lineage queries of the provenance.

We feel that these packages make a novel contribution to the R ecosystem as they allow for easy collection, examination, and sharing of provenance that is not currently available to R programmers and would be of interest to readers of the R Journal.

Thank you for your time and consideration.

We look forward to your reply.

\vspace{2\parskip} % Extra whitespace for aesthetics
\closing{Sincerely,}
\vspace{2\parskip} % Extra whitespace for aesthetics

%\ps{P.S. You can find additional information attached to this letter.} % Postscript text, comment this line to remove it

%\encl{Copyright permission form} % Enclosures with the letter, comment this line to remove it

%----------------------------------------------------------------------------------------

\end{letter}
 
\end{document}
