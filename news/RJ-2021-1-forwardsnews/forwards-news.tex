% !TeX root = RJwrapper.tex
\title{News from the Forwards Taskforce}
\author{by Heather Turner}

\maketitle

\href{https://forwards.github.io/}{Forwards} is an R Foundation taskforce working to widen the participation of under-represented groups in the R project and in related activities, such as the \emph{useR!} conference. This report rounds up activities of the taskforce during the first half of 2021.

\section{Community engagement}

Kevin O'Brien has been conducting a series of video interviews as part of the \emph{Why R? World} series. A number of interviews are ready to watch on the \href{https://www.youtube.com/playlist?list=PLKMUlj_pGn_mpO0B7eH9ao6SbPg0eLWLG}{Why R? Foundation} YouTube channel and several more rehearsal interviews have been conducted preparing the ground for a later recording. The interviewees include organizers of local groups, such as Nontsikelelo Shongwe (Eswatini); other community-builders, such as Reinaldo Zazela (Mozambique) connecting R users in Lusophone Africa, and other data scientists/related professionals talking about their work, such as Lais Carvalho (Brazil/Ireland) working in Developer Relations.

Kevin also facilitated the \href{https://www.youtube.com/c/WhyRFoundationVideos/search?query=\%22R-Ladies\%20Remote\%20Takeover\%22}{R-Ladies Remote Takeover} of the \emph{Why R? Webinar} series. Co-organized with Janani Ravi (R-Ladies Remote/R-Ladies East Lansing) and Heather Turner (Forwards/R-Ladies Remote) the \emph{Takeover} featured three talks from speakers from across the globe, spread throughout one day. First up was Afiqah Masrani (Malaysia) on \emph{Using \{gtsummary\} For Public Health Research}, then Tuli Amutenya (Namibia), on \emph{Natural Language Processing for Survey Text Data} and finally Beatriz Valdez (Venezuela) on \emph{Using R and Text Mining to find a Common Ground for Action in Polarized Contexts}. We were happy to provide a platform for these R-Ladies who live far from an active group (both \href{https://www.meetup.com/rladies-caracas/}{R-Ladies} and \href{https://twitter.com/GrupoRVenezuela}{R User} groups are being started in Caracas, Venezuela, however). The speaker from each talk co-chaired the next talk helping to make connections and a good atmosphere for the Q\&A. We hope to run a similar event in future.

\section{R Contribution Working Group}

The \href{https://forwards.github.io/rcontribution/working-group}{R Contribution Work Group} has begun work on some initiatives to encourage new contributors to R core, having laid the groundwork in the second half of 2020.

The R Foundation funded a 10-week project to develop a first version of the \href{https://forwards.github.io/rdevguide/}{R Developer's Guide}, a user-friendly introduction to contributing to R core. This project was undertaken by Saranjeet Kaur, mentored by Michael Lawrence and Heather Turner. Currently the guide covers identifying, reporting and reviewing bugs; preparing and submitting a patch; contributing to documentation, and testing pre-release versions of R. It also includes some technical help (building and installing R-devel on Windows, developer tools) and some community orientation (list of R core developers/contributors, where to get help and keep up-to-date with R project news). This guide benefited from review by RCWG members during and after the project, and we welcome members of the wider R developer community to review the guide and contribute to its further development as documented in \href{https://forwards.github.io/rdevguide/introduction.html#how-to-contribute-to-this-guide}{Chapter 1} of the guide.

The \href{https://forwards.github.io/rcontribution/slack}{R-devel Slack} group now has over 100 members and is gradually seeing more activity. We welcome anyone interested in contributing to R to join - it provides a space for wider discussion compared to the R-devel mailing list and a supportive community for people new to contributing.

\section{\emph{useR! 2021}}

Forwards was involved in a wide range of activities related to \emph{useR! 2021}. 

The Conference Team provided advice to the organizers on aspects such as code of conduct and events for first-timers. Liz Hare was heavily involved with supporting accessibility practices, including helping to develop the accessibility guidelines for presenters, providing input to the communications team and testing the accessibility of conference tools. The latter led to some improvements for screen-reader users in \href{https://thelounge.chat/}{The Lounge} chat platform; although this tool was ultimately abandoned for \emph{useR!}, these changes will benefit future users. Noa Tamir began work as a Senior Writer/Editor on the \href{https://github.com/rstats-gsod/gsod2021/wiki/GSOD-2021-Proposal}{R Project's Google Season of Docs} project to develop documentation supporting \emph{useR!} organization.

The Community Team helped with encouraging participation from under-represented countries, directly contacting R users to let them know about \emph{useR!} and the fee waivers available for those without funding. They also helped to enlist reviewers for \href{https://mircommunity.com/}{MiR}'s pre-review service, as well as Zoom hosts and volunteers for the team providing online chat support during the conference.

The Survey Team supported the organizers in running a Diversity Survey, the preliminary results of which were presented in the Closing Session.

The On-ramps Team partnered with the R Contribution Working Group to arrange two contributor-focused tutorials. The first was on \emph{Translating R to Your Language}, lead by Michael Chirico in collaboration with Michael Lawrence. This was attended by an enthusiastic group, that worked on translating R messages into Spanish, Bahasa Indonesia, Hindi and Hungarian. The second was on \emph{Contributing to R} lead by Gabriel Becker in collaboration with Martin Mächler. Gabriel shared his experiences as an external contributor to R core, before the participants split into small groups to debug a past issue in R 3.3.2. Martin reviewed how the bug was fixed in later versions of R and the session was completed with some parting advice and a round table, with Michael Lawrence joining as a guest. The participants reported that the tutorial was accessible and rewarding. A sub-group has met since to work through some of the further exercises prepared by the tutors.

Jonathan Godfrey, long-time member of Forwards, was part of a group keynote on responsible programming. He touched on the importance of acknowledging disabled people in our communities and using appropriate language when speaking about disability. However, his main theme was choosing tools that are inclusive. "To be disabled means you have been excluded in some way. But when I am included, I am no longer disabled." An example is that HTML documentation is much more accessible to screen-readers than PDF. As a community/developers, we should aim to "make the right things easy to do, and the incorrect ones, harder". 

The tutorials and sessions mentioned above are expected to be published soon on the \href{https://www.youtube.com/channel/UC_R5smHVXRYGhZYDJsnXTwg}{R Consortium YouTube channel} - links will also be shared on the \href{https://user2021.r-project.org/}{\emph{useR! 2021} website}.

\section{Package Development Modules}

As reported in the last issue, the Teaching Team have been modularizing the Forwards Package Development workshop. The first 3 modules were run in February 2021, with a total of 51 people. Only 19 people took all three modules, suggesting the modular approach enables participants to select modules  according to their existing knowledge.

The modules will be re-run in September, with an additional module on documentation and testing. The modules will be run on separate days: 20, 21, 23 and 24 September at 13:00 UTC. Registration pages for each module can be found from the \href{https://github.com/forwards/website_source/pull/119/commits/6e7652ba5190d8da7fd7caefb20e938f64423c7f}{Forwards Eventbrite Page}.

\address{Heather Turner\\
  University of Warwick, UK\\
  \email{Heather.Turner@R-project.org}}