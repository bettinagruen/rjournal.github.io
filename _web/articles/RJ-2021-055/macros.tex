\usepackage[utf8]{inputenc}

%-----------------------
% packages for (hyper-)links
%-----------------------
\usepackage{hyperref}
\hypersetup{
    colorlinks=true,
    linkcolor=blue,
    filecolor=blue,      
    citecolor=.,
    urlcolor=blue,
}
\usepackage{url}
\usepackage{catchfilebetweentags}
%-----------------------
% packages for numbering figures
%-----------------------
\usepackage{chngcntr}
%-----------------------
% packages for algorithm pseudocode
%-----------------------
\usepackage{algpseudocode,algorithm,algorithmicx}
\algnewcommand\algorithmicswitch{\textbf{switch}}
\algnewcommand\algorithmiccase{\textbf{case}}
\algnewcommand\algorithmicassert{\texttt{assert}}
\algnewcommand\Assert[1]{\State \algorithmicassert(#1)}%
% New "environments"
\algdef{SE}[SWITCH]{Switch}{EndSwitch}[1]{\algorithmicswitch\ #1\ \algorithmicdo}{\algorithmicend\ \algorithmicswitch}%
\algdef{SE}[CASE]{Case}{EndCase}[1]{\algorithmiccase\ #1}{\algorithmicend\ \algorithmiccase}%
\algtext*{EndSwitch}%
\algtext*{EndCase}%
\let\oldReturn\Return
\renewcommand{\Return}{\State\oldReturn}
%-----------------------
% mathematical typesetting packages
%-----------------------
\usepackage{mathtools}
\usepackage{amsmath,amssymb,amsthm}
\usepackage{thmtools}
\usepackage{bm}

%-----------------------
% graphics packages
%-----------------------
\usepackage{graphicx}
	\graphicspath{{images/}}
\usepackage{pifont}

%-----------------------
% footnotes
%-----------------------
\usepackage{todonotes}
%\renewcommand{\thefootnote}{\fnsymbol{footnote}}
\usepackage{authblk}
\usepackage[perpage, bottom, ragged]{footmisc}
%\renewcommand{\thefootnote}{\fnsymbol{footnote}}
\newcommand{\footurl}[1]{\footnote{\url{#1}}}

%-----------------------
% font packages
%-----------------------
%\usepackage[charter]{mathdesign}
\usepackage[mathcal]{eucal}
%\usepackage{mathptmx}
\usepackage{bbm}

%-----------------------
% text formatting packages
%-----------------------
%\usepackage{setspace}
\usepackage[inline, shortlabels]{enumitem}

% define quick list styles
% inline
\newlist{lsti}{enumerate*}{1}
\setlist[lsti]{label=\protect\litem\roman*)}
\newlist{lstI}{enumerate*}{1}
\setlist[lstI]{label=\protect\litem\Roman*)}
\newlist{lsta}{enumerate*}{1}
\setlist[lsta]{label=\protect\litem\alph*)}
\newlist{lstA}{enumerate*}{1}
\setlist[lstA]{label=\protect\litem\Alph*)}
\newlist{lstn}{enumerate*}{1}
\setlist[lstn]{label=\protect\litem\arabic*)}

\newcommand{\atem}{\global\andtemtrue\item}
\newcommand{\otem}{\global\ortemtrue\item}
\newcommand{\litem}{%
  \ifandtem and \global\andtemfalse\fi
  \ifortem or \global\ortemfalse\fi
}
\newif\ifandtem
\newif\ifortem

\usepackage[framemethod=tikz]{mdframed}
\theoremstyle{definition}
\newmdtheoremenv[%
    innerlinewidth=0.5pt,%
    innerleftmargin=10pt,%
    innerrightmargin=10pt,%
    innertopmargin = 0pt,%
    innerbottommargin=5pt,%
    roundcorner=1pt%
]{nbox}{Box}
	
	

%-----------------------
% page formatting packages
%-----------------------
\usepackage{pdflscape}
\usepackage{afterpage}
\usepackage{setspace}
\usepackage{geometry}

%-----------------------
% packages for table building
%-----------------------
\usepackage{multirow}
\usepackage{array}
\usepackage{tabularx}
 \newcolumntype{L}{>{\raggedright\arraybackslash}X}
\usepackage{makecell}
\usepackage{booktabs}
\usepackage[font=small,labelfont=bf]{caption}
\setlength{\heavyrulewidth}{1.5pt}
\setlength{\abovetopsep}{4pt}
\usepackage{diagbox}
\usepackage{threeparttable}
\usepackage{threeparttablex}
%\usepackage{rotating}
\usepackage{longtable}

%-----------------------
% packages for figure building
%-----------------------
\usepackage{float}
\usepackage[caption = false]{subfig}

%-----------------------
% citation, referencing, and bibliography packages
%-----------------------
%\usepackage[square,numbers,sort&compress]{natbib}
%\usepackage[numbers]{natbib}
\usepackage{authblk}
\usepackage[backend=bibtex]{biblatex}

%-----------------------
% section header settings
%-----------------------
\usepackage{titlesec}
\titleformat{\section}
	{\normalfont\Large\bfseries\filcenter}{\thesection.}{1 ex}{}
\titleformat{\subsection}%[runin]
	{\normalfont\large\bfseries}{\thesubsection.}{1 ex}{}
\titleformat{\subsubsection}%[runin]
	{\normalfont\normalsize\bfseries}{\thesubsubsection.}{1 ex}{}
	
\titleformat{\chapter}[display]   
{\normalfont\huge\bfseries}{\chaptertitlename\ \thechapter}{20pt}{\Huge}
	
\usepackage{fancyhdr}


%-----------------------
%table of content and section number depth

% 0 = chapters only
% 1 = sections appear
% 2 = subsections appear
% 3 = subsubsections appear
% 4 = subsubsubsections appear
%-----------------------
\setcounter{tocdepth}{1}
\setcounter{secnumdepth}{3}

%-----------------------
%thmtools declarations
%-----------------------
\declaretheorem[name=Theorem]{Thm}
\declaretheorem[within=section,name=Lemma]{Lem}
\declaretheorem[sibling=Lem,name=Definition]{Def}
\declaretheorem[sibling=Lem,name=Assumption]{Ass}
\declaretheorem[sibling=Lem,name=Notation]{Not}
\declaretheorem[sibling=Lem,name=Proposition]{Prop}
\declaretheorem[sibling=Lem,name=Remark]{Rem}
\declaretheorem[sibling=Lem,name=Example, qed=$\square$]{Ex}
\declaretheorem[sibling=Lem,name=Corollary]{Cor}
\declaretheorem[sibling=Thm,name=Conjecture]{Conj}

%-----------------------
% other declarations
%-----------------------
\numberwithin{equation}{section}
%\numberwithin{figure}{section}
\theoremstyle{plain}
\newtheorem{thm}{\protect\theoremname}
\numberwithin{thm}{section}
  \theoremstyle{definition}
  \newtheorem{example}[thm]{\protect\examplename}
  \theoremstyle{definition}
  \newtheorem{defn}[thm]{\protect\definitionname}
  \theoremstyle{plain}
  \newtheorem{prop}[thm]{\protect\propositionname}
  \numberwithin{prop}{section}
  \theoremstyle{plain}
  \newtheorem{lem}[thm]{\protect\lemmaname}
  \numberwithin{lem}{section}
  \theoremstyle{remark}
  \newtheorem{rem}[thm]{\protect\remarkname}
  \theoremstyle{plain}
  \newtheorem{cor}[thm]{\protect\corollaryname}

\newtheorem{conj}[thm]{Conjecture}

\usepackage{babel}
  \providecommand{\corollaryname}{Corollary}
  \providecommand{\definitionname}{Definition}
  \providecommand{\examplename}{Example}
  \providecommand{\lemmaname}{Lemma}
  \providecommand{\remarkname}{Remark}
  \providecommand{\theoremname}{Theorem}
  \providecommand{\propositionname}{Proposition}

\newtheorem*{remark}{Remark}

\theoremstyle{definition}
\newtheorem{definition}{Definition}[section]

\newenvironment{subproof}[1][\proofname]{%
  \renewcommand{\qedsymbol}{$\blacksquare$}%
  \begin{proof}[#1]%
}{%
  \end{proof}%
}

%-----------------------
% self-reference
%-----------------------
\usepackage{cleveref}
\crefname{lem}{lemma}{lemmas}
\Crefname{lem}{Lemma}{Lemmas}
\crefname{thm}{theorem}{theorems}
\Crefname{thm}{Theorem}{Theorems}
\crefname{cor}{corollary}{corollaries}
\Crefname{cor}{Corollary}{Corollaries}
\crefname{prop}{proposition}{propositions}
\Crefname{prop}{Proposition}{Propositions}
\crefname{lstlisting}{listing}{listings}
\Crefname{lstlisting}{Listing}{Listings}
\crefname{conj}{conjecture}{conjectures}
\Crefname{conj}{Conjecture}{Conjectures}
\crefname{nbox}{box}{boxes}
\Crefname{nbox}{Box}{Boxes}

\newcommand{\pr}[1]{(p. \pageref{#1})}

%-----------------------
% frequently used mathematical macros
%-----------------------
\newcommand{\eop}{\hfill$\qed$}

\renewcommand{\vec}[1]{\mathbf{#1}}
\newcommand{\cb}{\mathbf{c}}
\newcommand{\mb}{\mathbf{m}}
\newcommand{\nb}{\mathbf{n}}

\newcommand{\varX}{\ensuremath{X}}
\newcommand{\varY}{\ensuremath{Y}}

\newcommand{\Frac}{\operatorname{Frac}}
\newcommand{\Stab}{\operatorname{Stab}}
\newcommand{\Spec}{\operatorname{Spec}}
\newcommand{\Supp}{\operatorname{Supp}}
\newcommand{\supp}{\operatorname{supp}}
\newcommand{\cosupp}{\operatorname{cosupp}}
\newcommand{\vsupp}{\operatorname{vsupp}}
\newcommand{\Ker}{\operatorname{Ker}}
\newcommand{\Gr}{\operatorname{Gr}}
\newcommand{\codim}{\operatorname{codim}}
\newcommand{\rk}{\operatorname{rk}}
\newcommand{\argmin}{\operatornamewithlimits{argmin}}
\newcommand{\argmax}{\operatornamewithlimits{argmax}}
\newcommand{\lcm}{\operatorname{lcm}}
\newcommand{\trdeg}{\operatorname{trdeg}}
\newcommand{\codeg}{\operatorname{codeg}}
\newcommand{\id}{\operatorname{id}}
\newcommand{\Tr}{\operatorname{Tr}}
\newcommand{\res}{\operatorname{res}}
\newcommand{\rank}{\operatorname{rank}}
\newcommand{\chr}{\operatorname{char}}
\newcommand{\sgn}{\operatorname{sgn}}
\newcommand{\mesh}{\operatorname{mesh}}
\newcommand{\Corr}{\operatorname{Corr}}
\newcommand{\Cov}{\operatorname{Var}}
\newcommand{\Var}{\operatorname{Var}}
\newcommand{\Bias}{\operatorname{Bias}}
\newcommand{\logit}{\operatorname{logit}}
\newcommand{\odds}{\operatorname{odds}}

\newcommand{\msupp}{\operatorname{msupp}}
\newcommand{\rowsupp}{\operatorname{rowsupp}}
\newcommand{\colsupp}{\operatorname{colsupp}}

\newcommand{\tvi}{\textrm{t.v.i.}\xspace}
\newcommand{\rv}{\textrm{r.v.}\xspace}
\newcommand{\st}{\textrm{s.t.}\xspace}
\newcommand{\iid}{\ensuremath{\stackrel{i.i.d.}\sim}}
\newcommand{\otherw}{\text{otherwise}}

\newcommand{\compresslist}{\setlength{\itemsep}{1pt}
  \setlength{\parskip}{0pt}
  \setlength{\parsep}{0pt} }

\newcommand{\spn}{\operatorname{span}}

\newcommand{\To}{\longrightarrow}

%\newcommand{\bigast}{\mathop{\Huge \mathlarger{\mathlarger{\ast}}}}


\newcommand{\card}{\#}
\DeclarePairedDelimiter{\ceil}{\lceil}{\rceil}
\DeclarePairedDelimiter\floor{\lfloor}{\rfloor}

\newcommand{\mean}[2][n]{\frac{1}{#1} \sum^{#1}_{i = 1} #2}

%-----------------------
% Capital Greek letters
%-----------------------
\newcommand{\Alpha}{\mathrm{A}}
\newcommand{\Beta}{\mathrm{B}}
\newcommand{\Epsilon}{\mathrm{E}}
\newcommand{\Tau}{\mathrm{T}}

%-----------------------
% calligraphic letters
%-----------------------
\newcommand{\calA}{\mathcal{A}}
\newcommand{\calB}{\mathcal{B}}
\newcommand{\calC}{\mathcal{C}}
\newcommand{\calD}{\mathcal{D}}
\newcommand{\calE}{\mathcal{E}}
\newcommand{\calF}{\mathcal{F}}
\newcommand{\calG}{\mathcal{G}}
\newcommand{\calH}{\mathcal{H}}
\newcommand{\calI}{\mathcal{I}}
\newcommand{\calJ}{\mathcal{J}}
\newcommand{\calK}{\mathcal{K}}
\newcommand{\calL}{\mathcal{L}}
\newcommand{\calM}{\mathcal{M}}
\newcommand{\calN}{\mathcal{N}}
\newcommand{\calO}{\mathcal{O}}
\newcommand{\calP}{\mathcal{P}}
\newcommand{\calQ}{\mathcal{Q}}
\newcommand{\calR}{\mathcal{R}}
\newcommand{\calS}{\mathcal{S}}
\newcommand{\calT}{\mathcal{T}}
\newcommand{\calU}{\mathcal{U}}
\newcommand{\calV}{\mathcal{V}}
\newcommand{\calW}{\mathcal{W}}
\newcommand{\calX}{\mathcal{X}}
\newcommand{\calY}{\mathcal{Y}}
\newcommand{\calZ}{\mathcal{Z}}

%-----------------------
% hat letters
%-----------------------
\newcommand{\hata}{\hat{a}}
\newcommand{\hatb}{\hat{b}}
\newcommand{\hatc}{\hat{c}}
\newcommand{\hatd}{\hat{d}}
\newcommand{\hate}{\hat{e}}
\newcommand{\hatf}{\hat{f}}
\newcommand{\hatg}{\hat{g}}
\newcommand{\hath}{\hat{h}}
\newcommand{\hati}{\hat{i}}
\newcommand{\hatj}{\hat{j}}
\newcommand{\hatk}{\hat{k}}
\newcommand{\hatl}{\hat{l}}
\newcommand{\hatm}{\hat{m}}
\newcommand{\hatn}{\hat{n}}
\newcommand{\hato}{\hat{o}}
\newcommand{\hatp}{\hat{p}}
\newcommand{\hatq}{\hat{q}}
\newcommand{\hatr}{\hat{r}}
\newcommand{\hats}{\hat{s}}
\newcommand{\hatt}{\hat{t}}
\newcommand{\hatu}{\hat{u}}
\newcommand{\hatv}{\hat{v}}
\newcommand{\hatw}{\hat{w}}
\newcommand{\hatx}{\hat{x}}
\newcommand{\haty}{\hat{y}}
\newcommand{\hatz}{\hat{z}}

\newcommand{\hatA}{\hat{A}}
\newcommand{\hatB}{\hat{B}}
\newcommand{\hatC}{\hat{C}}
\newcommand{\hatD}{\hat{D}}
\newcommand{\hatE}{\hat{E}}
\newcommand{\hatF}{\hat{F}}
\newcommand{\hatG}{\hat{G}}
\newcommand{\hatH}{\hat{H}}
\newcommand{\hatI}{\hat{I}}
\newcommand{\hatJ}{\hat{J}}
\newcommand{\hatK}{\hat{K}}
\newcommand{\hatL}{\hat{L}}
\newcommand{\hatM}{\hat{M}}
\newcommand{\hatN}{\hat{N}}
\newcommand{\hatO}{\hat{O}}
\newcommand{\hatP}{\hat{P}}
\newcommand{\hatQ}{\hat{Q}}
\newcommand{\hatR}{\hat{R}}
\newcommand{\hatS}{\hat{S}}
\newcommand{\hatT}{\hat{T}}
\newcommand{\hatU}{\hat{U}}
\newcommand{\hatV}{\hat{V}}
\newcommand{\hatW}{\hat{W}}
\newcommand{\hatX}{\hat{X}}
\newcommand{\hatY}{\hat{Y}}
\newcommand{\hatZ}{\hat{Z}}

%-----------------------
% fracture letters
%-----------------------
\newcommand{\frakA}{\mathfrak{A}}
\newcommand{\frakI}{\mathfrak{I}}
\newcommand{\frakJ}{\mathfrak{J}}
\newcommand{\frakP}{\mathfrak{P}}
\newcommand{\frakQ}{\mathfrak{Q}}
\newcommand{\frakS}{\mathfrak{S}}
\newcommand{\frakY}{\mathfrak{Y}}

\newcommand{\fraks}{\mathfrak{s}}

%-----------------------
% double lined letters
%-----------------------
\newcommand{\ZZ}{\ensuremath{\mathbb{Z}}}
\newcommand{\RR}{\ensuremath{\mathbb{R}}}
\newcommand{\PP}{\ensuremath{\mathbb{P}}}
\newcommand{\QQ}{\ensuremath{\mathbb{Q}}}
\newcommand{\FF}{\ensuremath{\mathbb{F}}}
\newcommand{\CC}{\ensuremath{\mathbb{C}}}
\newcommand{\NN}{\ensuremath{\mathbb{N}}}
\newcommand{\KK}{\ensuremath{\mathbb{K}}}
\newcommand{\EE}{\ensuremath{\mathbb{E}}}
\newcommand{\II}{\ensuremath{\mathbb{I}}}
\newcommand{\xx}{\ensuremath{\mathbf{x}}}

%-----------------------
% number sets
%-----------------------
\newcommand*{\Naturals}{\ensuremath{\mathbb{N}_0}}
\newcommand*{\PNaturals}{\ensuremath{\mathbb{N}_{> 0}}}

\newcommand*{\Integers}{\ensuremath{\mathbb{Z}}}
\newcommand*{\PIntegers}{\ensuremath{\mathbb{Z}_{>0}}}
\newcommand*{\NNIntegers}{\ensuremath{\mathbb{Z}_{\geq 0}}}
\newcommand*{\NIntegers}{\ensuremath{\mathbb{Z}_{< 0}}}
\newcommand*{\NPIntegers}{\ensuremath{\mathbb{Z}_{\leq 0}}}

\newcommand*{\Rationals}{\ensuremath{\mathbb{Q}}}
\newcommand*{\PRationals}{\ensuremath{\mathbb{Q}_{>0}}}
\newcommand*{\NNRationals}{\ensuremath{\mathbb{Q}_{\geq 0}}}
\newcommand*{\NRationals}{\ensuremath{\mathbb{Q}_{< 0}}}
\newcommand*{\NPRationals}{\ensuremath{\mathbb{Q}_{\leq 0}}}

\newcommand*{\Reals}{\ensuremath{\mathbb{R}}}
\newcommand*{\PReals}{\ensuremath{\mathbb{R}_{>0}}}
\newcommand*{\NNReals}{\ensuremath{\mathbb{R}_{\geq 0}}}
\newcommand*{\NReals}{\ensuremath{\mathbb{R}_{< 0}}}
\newcommand*{\NPReals}{\ensuremath{\mathbb{R}_{\leq 0}}}

\newcommand*{\ExtReals}{\ensuremath{\bar{\mathbb{R}}}}
\newcommand*{\Complex}{\ensuremath{\mathbb{C}}}

\newcommand{\bset}{\{0,1\}}
%-----------------------
% double lined numbers
%-----------------------
\newcommand{\OOne}{\ensuremath{\mathbbm{1}}}

%-----------------------
% bold letters
%-----------------------
\newcommand{\bfi}{\mathbf{i}}
\newcommand{\bft}{\mathbf{t}}

\newcommand{\ba}{\boldsymbol{a}}
\newcommand{\bb}{\boldsymbol{b}}
\newcommand{\bo}{\boldsymbol{o}}
\newcommand{\bi}{\boldsymbol{i}}
\newcommand{\bj}{\boldsymbol{j}}
\newcommand{\bp}{\boldsymbol{p}}
\newcommand{\br}{\boldsymbol{r}}
\newcommand{\bs}{\boldsymbol{s}}
\newcommand{\bt}{\boldsymbol{t}}
\newcommand{\bu}{\boldsymbol{u}}
\newcommand{\bw}{\boldsymbol{w}}
\newcommand{\bx}{\boldsymbol{x}}
\newcommand{\by}{\boldsymbol{y}}

\newcommand{\balpha}{\boldsymbol{\alpha}}
\newcommand{\bbeta}{\boldsymbol{\beta}}
\newcommand{\bgamma}{\boldsymbol{\gamma}}
\newcommand{\blambda}{\boldsymbol{\lambda}}
\newcommand{\bsigma}{\boldsymbol{\sigma}}
\newcommand{\btau}{\boldsymbol{\tau}}

\newcommand{\bL}{\boldsymbol{L}}
\newcommand{\bN}{\boldsymbol{N}}
\newcommand{\bP}{\boldsymbol{P}}
\newcommand{\bW}{\boldsymbol{W}}
\newcommand{\bX}{\boldsymbol{X}}
\newcommand{\bY}{\boldsymbol{Y}}

%-----------------------
% distributions
%-----------------------
\newcommand{\Distr}{\operatorname{Distr}}
\newcommand{\Bern}{\operatorname{Bern}}
\newcommand{\Binom}{\operatorname{Binom}}
\newcommand{\Exp}{\operatorname{Exp}}
\newcommand{\Gomp}{\operatorname{Gompertz}}
\newcommand{\Hyper}{\operatorname{Hyper}}
\newcommand{\Llogis}{\operatorname{Loglogistic}}
\newcommand{\Lnorm}{\operatorname{Lognormal}}
\newcommand{\Norm}{\ensuremath{\calN}}
\newcommand{\Unif}{\ensuremath{\calU}}
\newcommand{\Weib}{\operatorname{Weibull}}

%-----------------------
%differential operator
%-----------------------
\makeatletter
\providecommand*{\diff}%
        {\@ifnextchar^{\DIfF}{\DIfF^{}}}
\def\DIfF^#1{%
        \mathop{\mathrm{\mathstrut d}}%
                \nolimits^{#1}\gobblespace
}
\def\gobblespace{%
        \futurelet\diffarg\opspace}
\def\opspace{%
        \let\DiffSpace\!%
        \ifx\diffarg(%
                \let\DiffSpace\relax
        \else
                \ifx\diffarg\[%
                        \let\DiffSpace\relax
                \else
                        \ifx\diffarg\{%
                                \let\DiffSpace\relax
                        \fi\fi\fi\DiffSpace}
\makeatother

%-----------------------
%indep and not indep
%-----------------------
\makeatletter
\newcommand*{\indep}{%
  \mathbin{%
    \mathpalette{\@indep}{}%
  }%
}
\newcommand*{\nindep}{%
  \mathbin{%                   % The final symbol is a binary math operator
    \mathpalette{\@indep}{\not}% \mathpalette helps for the adaptation
                               % of the symbol to the different math styles.
  }%
}
\newcommand*{\@indep}[2]{%
  % #1: math style
  % #2: empty or \not
  \sbox0{$#1\perp\m@th$}%        box 0 contains \perp symbol
  \sbox2{$#1=$}%                 box 2 for the height of =
  \sbox4{$#1\vcenter{}$}%        box 4 for the height of the math axis
  \rlap{\copy0}%                 first \perp
  \dimen@=\dimexpr\ht2-\ht4-.2pt\relax
      % The equals symbol is centered around the math axis.
      % The following equations are used to calculate the
      % right shift of the second \perp:
      % [1] ht(equals) - ht(math_axis) = line_width + 0.5 gap
      % [2] right_shift(second_perp) = line_width + gap
      % The line width is approximated by the default line width of 0.4pt
  \kern\dimen@
  {#2}%
      % {\not} in case of \nindep;
      % the braces convert the relational symbol \not to an ordinary
      % math object without additional horizontal spacing.
  \kern\dimen@
  \copy0 %                       second \perp
} 
\makeatother

%-----------------------
% listings
%-----------------------
\usepackage{listings}
\usepackage{color}
\definecolor{codegreen}{rgb}{0,0.6,0}
\definecolor{codegray}{rgb}{0.5,0.5,0.5}
\usepackage{textcomp}
\lstdefinestyle{mystyle}{
    commentstyle=\bfseries\color{codegreen},
    keywordstyle=\color{black},
    numberstyle=\tiny\color{codegray},
    stringstyle=\color{black},
    basicstyle=\ttfamily,
    breakatwhitespace=false,         
    breaklines=true,                 
   % captionpos=b,                    
    keepspaces=true,              
    showspaces=false,                
    showstringspaces=false,
    showtabs=false,                  
    tabsize=2,
    language = R,
	alsoletter={\$},
	morekeywords={decorate}	
  identifierstyle=\color{black},
}

\lstset{upquote=true, escapeinside={(*@}{@*)}}

\usepackage{xcolor}
\usepackage{listings}
\usepackage{color}
\usepackage{textcomp}

%-----------------------
% tikz
%-----------------------
\usepackage{fancyvrb}
%-----------------------
% pgfgantt
%-----------------------
\usepackage{pgfgantt}

%-----------------------
% misc
%-----------------------
\usepackage{xspace}
\newcommand{\hl}[1]{\textbf{\textcolor{red}{#1}}}
\newcommand{\code}[1]{\texttt{#1}\xspace}
\newcommand{\pkg}[1]{\textbf{#1}\xspace}
\newcommand{\Rstats}{\textsf{R}\xspace}
\newcommand{\see}[2][]{(\cref{#2}#1)\xspace}
\newcommand{\etal}{\textit{et al.}\xspace}

%-----------------------
% appendices
%-----------------------
\usepackage[toc,page]{appendix}

%-----------------------
% csquotes
%-----------------------
\usepackage{csquotes}
\renewcommand{\mkbegdispquote}[2]{\itshape}