% !TeX root = RJwrapper.tex
\title{The R Developer Community Does Have a Strong Software Engineering
Culture}
\author{by Maëlle Salmon and Karthik Ram}

\maketitle

\abstract{%
There is a strong software engineering culture in the R developer
community. We recommend creating, updating and vetting packages as well
as keeping up with community standards. We invite contributions to the
rOpenSci project, where participants can gain experience that will shape
their work and that of their peers.
}

\hypertarget{introduction}{%
\subsection{Introduction}\label{introduction}}

The R programming language was originally created for statisticians, by
statisticians, but evolved over time to attract a ``massive pool of
talent that was previously untapped'' (Hadley Wickham in
\citet{rgeneration}). Despite the fact that most R users are academic
researchers and business data analysts without a background in software
engineering, we are witnessing a rapid rise in software engineering
within the community. In this comment we spotlight recent progress in
tooling, dissemination and support, including specific efforts led by
the rOpenSci project. We hope that readers will take advantage of and
participate in the tools and practices we describe.

\hypertarget{the-modern-r-package-developer-toolbox-user-friendlier-more-comprehensive}{%
\subsection{The modern R package developer toolbox: user-friendlier,
more
comprehensive}\label{the-modern-r-package-developer-toolbox-user-friendlier-more-comprehensive}}

The basic infrastructure for creating, building, installing, and
checking packages has been in place since the early days of the R
language. During this time (1998-2011), the barriers to entry were very
high and access to support and Q\&A for beginners were extremely
limited. With the introduction of the \CRANpkg{devtools}
\citep{devtools} package in 2011, the process of creating and updating
packages became substantially easier. Documentation also became simpler
to maintain. The \CRANpkg{roxygen2} \citep{roxygen2} package allowed
developers to keep documentation in sync with changes in code, similar
to the doxygen approach that was embraced in more mature languages.
Combined with the rise in popularity of StackOverflow and the growth of
rstats blogs, the number of packages on the Comprehensive R Archive
Network (CRAN) skyrocketed from 400 new packages in 2010 to 1000 new
packages by 2014. As of this writing, there are nearly 19k packages on
CRAN.

For novices without substantial software engineer experience, the early
testing frameworks were also difficult to use. With the release of
\CRANpkg{testthat} \citep{testthat}, testing also became smoother. There
are now several actively maintained testing frameworks such as
\CRANpkg{tinytest} \citep{tinytest}; as well as testthat-compatible
specialized tooling for testing database interactions (\CRANpkg{dittodb}
\citep{dittodb}), web resources (\CRANpkg{vcr} \citep{vcr}),
\CRANpkg{httptest} \citep{httptest}, and \CRANpkg{webfakes}
\citep{webfakes} which enables the use of an embedded C/C++ web server
for testing HTTP clients like \CRANpkg{httr2} \citep{httr2}).

The testthat package has recently been improved with snapshot tests that
make it possible to test plot outputs. The rOpenSci project has released
\CRANpkg{autotest} \citep{autotest}, a package that supports automatic
mutation testing.

Beyond checking for compliance with R CMD CHECK, several other packages
such as \CRANpkg{goodpractice} \citep{goodpractice},
\CRANpkg{riskmetric} \citep{riskmetric}, rOpenSci's \CRANpkg{pkgcheck}
\citep{pkgcheck} check packages against a large list of actionable,
community recommended best practices for software development.
Collectively these tools allow domain researchers to release software
packages that meet high standards for software engineering.

The development and testing ecosystem of R is rich and has sometimes
borrowed successful implementations from other languages (e.g.~the vcr R
package is a port, i.e.~translation to R, of the vcr Ruby gem; testthat
snapshot tests were inspired by JS Jest\footnote{\url{https://www.tidyverse.org/blog/2020/10/testthat-3-0-0/\#snapshot-testing}}).

\hypertarget{emergence-of-a-welcoming-community}{%
\subsection{Emergence of a welcoming
community}\label{emergence-of-a-welcoming-community}}

As underlined in \citet{rgeneration}, community is the strong suit of
the R language. Many organizations and venues offer dedicated support
for package developers. Examples include Q\&A on the r-package-devel
mailing list\footnote{\url{https://stat.ethz.ch/mailman/listinfo/r-package-devel}},
and the package development category of the RStudio community
forum\footnote{\url{https://community.rstudio.com/c/package-development/11}},
and the rstats section of StackOverflow\footnote{\url{https://stackoverflow.com/questions/tagged/r?tab=Newest}}.
Traditionally, R package developers have been mostly male and white.
Although the status quo remains similar, efforts from groups such as
R-Ladies\footnote{\url{http://rladies.org/}} meetups, Minorities in R
\citep{mir}, and the package development modules offered by Forwards for
underrepresented groups\footnote{\url{https://buzzrbeeline.blog/2021/02/09/r-forwards-package-development-modules-for-women-and-other-underrepresented-groups/}}
have made considerable inroads towards improving diversity. These
efforts have worked hard to put the spotlight on developers beyond the
``usual suspects''.

\hypertarget{ropensci-community-and-software-review}{%
\subsection{rOpenSci community and software
review}\label{ropensci-community-and-software-review}}

The rOpenSci organization \citep{building} is an attractive venue for
developers \& supporters of scientific R software. One of our most
successful and continuing initiatives is our Software Peer Review system
\citep{cop}, a combination of academic peer-review and code review from
industry. About 150 packages have been reviewed by volunteers to date,
creating better packages as well as a growing knowledgebase in our
development guide \citep{devguide} while also building a living
community of practice.\\
Our model has been the fundamental inspiration for projects such as the
Journal of Open Source Software \citep{Smith2018}, and PyOpenSci
{[}\citet{pyopensci}{]}\citep{pyopensci2}. We are continuously improving
our system and reducing cognitive overload on editors and reviewers by
automating repetitive tasks. Most recently we have expanded our
offerings to peer review of packages that implement statistical methods
(Statistical Software Peer Review) \citep{mark_padgham_2021_5556756}.\\
Beside software review, rOpenSci community is a safe, welcoming and
informative place for package developers, with Q\&A happening on our
public forum and semi-open Slack workspace. \citep{ctbguide}

\hypertarget{creation-and-dissemination-of-resources-for-r-programmers}{%
\subsection{Creation and dissemination of resources for R
programmers}\label{creation-and-dissemination-of-resources-for-r-programmers}}

The aforementioned tools, venues and organizations benefit from and
support crucial dissemination efforts.\\
Publishing technical know-how is crucial for progress of the R
community. R news has been circulating on Twitter\footnote{\url{https://www.t4rstats.com/}},
R Weekly\footnote{\url{https://rweekly.org/}} and R-Bloggers\footnote{\url{https://www.r-bloggers.com/}}.
Some sources have been more specifically aimed at R package developers
of various experience and interests. While ``Writing R Extensions''
\footnote{\url{https://cran.r-project.org/doc/manuals/R-exts.html}} is
the official \& exhaustive reference on writing R packages, it is a
reference rather than a learning resource: many R package developers, if
not learning by example, get introduced to R package development via
introductory blog posts or tutorials, and the R packages book by Hadley
Wickham and Jenny Bryan {[}\citet{rpkgs}{]}\citep{rpkgs2} that accompany
the devtools suite of packages is freely available online and strives to
improving the R package development experience. The rOpenSci guide
``rOpenSci Packages: Development, Maintenance, and Peer Review''
\citep{devguide} contains our community-contributed guidance on how to
develop packages and review them. It features \emph{opinionated
requirements} such as the use of \CRANpkg{roxygen2} \citep{roxygen2} for
package documentation; \emph{criteria helping make an informed decision}
on gray area topics such as limiting dependencies; \emph{advice on
widely accepted and emerging tools}. As it is a living document also
used as reference for editorial decisions, we maintain a
changelog\footnote{\url{https://devguide.ropensci.org/booknews.html}},
and summarize each release in a blog post\footnote{\url{https://ropensci.org/tags/dev-guide/}}.
rOpenSci also hosts a book on a specialized topic, HTTP testing in
R\footnote{\url{https://books.ropensci.org/http-testing/}}, that
presents both principles for testing packages that interact with web
resources, as well as relevant packages. Beside these examples of
long-form documentation, knowledge around R software engineering is
shared through blogs and talks. In the R blogging world, the rOpenSci
blog posts\footnote{\url{https://ropensci.org/blog/}}, technical
notes\footnote{\url{https://ropensci.org/technotes/}} and a section of
our monthly newsletter\footnote{\url{https://ropensci.org/news/}}
feature some topics relevant to package developers, as do some of the
posts on the Tidyverse blog\footnote{\url{https://www.tidyverse.org/categories/programming/}}.
The blog of the R-hub project\footnote{\url{https://blog.r-hub.io/post/}}
contains information on package development topics, in particular about
common problems such as sharing data via R packages or understanding
CRAN checks. Expert programmers have been sharing their R specific
wisdom as well as software engineering lessons learned from other
languages (e.g.~Jenny Bryan's useR! Keynote address ``code feels, code
smells''\footnote{\url{https://github.com/jennybc/code-smells-and-feels}}).

\hypertarget{conclusion}{%
\subsection{Conclusion}\label{conclusion}}

In summary, we observe that there is already a strong software
engineering culture in the R developer community. By surfacing the rich
suite of resources to new developers we can but only hope the future
will bring success to all aforementioned initiatives. We recommend
creating, updating and vetting packages with the tools we mentioned as
well as keeping up with community standards with the venues we mentioned
in the previous section. We invite contributions to the rOpenSci
project, where participants can gain experience that will shape their
work and that of their peers. Thanks to these efforts, we hope the R
community will continue to be a thriving place of application for
software engineering, by diverse practitioners from many different
paths.

\bibliography{salmon-ram.bib}

\address{%
Maëlle Salmon\\
The rOpenSci Project\\%
\\ \\
%
\url{https://masalmon.eu}\\%
\textit{ORCiD: \href{https://orcid.org/0000-0002-2815-0399}{0000-0002-2815-0399}}\\%
\href{mailto:msmaellesalmon@gmail.com}{\nolinkurl{msmaellesalmon@gmail.com}}%
}

\address{%
Karthik Ram\\
Berkeley Institute for Data Science and The rOpenSci Project\\%
\\ \\
%
\url{https://ram.berkeley.edu/}\\%
\textit{ORCiD: \href{https://orcid.org/0000-0002-0233-1757}{0000-0002-0233-1757}}\\%
\href{mailto:karthik.ram@berkeley.edu}{\nolinkurl{karthik.ram@berkeley.edu}}%
}
