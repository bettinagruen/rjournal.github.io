% !TeX root = RJwrapper.tex
\title{Towards a Grammar for Processing Clinical Trial Data}
\author{by Michael J. Kane}

\maketitle

\abstract{%
The goal of this paper is to help define a path toward a grammar for
processing clinical trials by a) defining a format in which we would
like to represent data from standardized clinical trial data b)
describing a standard set of operations to transform clinical trial data
into this format, and c) to identify a set of verbs and other
functionality to facilitate data processing and encourage
reproducibility in the processing of these data. It provides a
background on standard clinical trial data and goes through a simple
preprocessing example illustrating the value of the proposed approach
through the use of the \pkg{forceps} package, which is currently being
used for data of this kind.
}

\hypertarget{introduction-on-the-use-of-historical-clinical-data}{%
\section{Introduction: On the use of historical clinical
data}\label{introduction-on-the-use-of-historical-clinical-data}}

There are few areas of data science research that provide more promise
to improve human quality-of-life and treat a disease than the
development of methods and analysis in clinical trials. While adjacent,
data-focused areas of biomedicine and health related research have
recently seen increased attention, especially the analysis of real-world
evidence (RWE) and electronic health records (EHR) in particular,
clinical trial data maintains several distinct quality advantages,
enumerated here.

\begin{enumerate}
\def\labelenumi{\arabic{enumi}.}
\tightlist
\item
  Features and measurements are selected for their relevance. Unlike
  EHR's or other similar data, variables collected for a clinical trial
  are included because they are potentially relevant to the disease
  under consideration or the treatment whose efficacy is being analyzed.
  This makes the variable selection process considerably easier than
  that where data collection has not been designed for a targeted
  analysis of this type.
\item
  Data collection procedures are carefully prescribed. Clinical trial
  data is uniform in both which variables are collected and how they are
  collected. This ensures data quality across trial sites ensuring that
  variables are relatively complete as well as consistent.
\item
  Inclusion/Exclusion criteria define the population. Since RWE studies
  are observational, the populations they consider are not always
  well-understood due to bias in the collection process. On the other
  hand, clinical trial data sets are generally controlled and
  randomized, with well-documented inclusion and exclusion criteria.
\end{enumerate}

Along with maintaining higher quality clinical trial data is more
available and more easily accessible when compared to real-world data
sources, which often require affiliations with appropriate research
institutions as well as infrastructure and appropriate staff, including
data managers, to extract data. By contrast, modern clinical trial data
organizations allow users to quickly search and download thousands of
trials, including anonymized patient-level information. These data sets
tend to include control-arm data, which can be used to understand
prognostic disease populations construct historical controls for
existing trials. However, some also include treatment data, which can be
used to characterize predictive patient subtypes for a given treatment,
understand safety profiles for classes of drugs, and aid in the design
of new trials. We note that, for oncology, Project Data Sphere
\citep{PDS} and outside of oncology, Immport \citep{Immport} have been
invaluable in our own experience by facilitating these types of
analyses.

\hypertarget{clinical-trial-analysis-data-sets}{%
\subsection{Clinical trial analysis data
sets}\label{clinical-trial-analysis-data-sets}}

During a clinical trial, patient-level data is collected in case report
forms (CRFs). The format and data collected in these forms are
prescribed in the trial design. These forms are the basis for the
construction of analysis data sets and other documents that will be
submitted to governing bodies, including the Food and Drug Association
(FDA) and European Medicines Agency (EMA), for approval if the sponsor
(party funding the trial) decides it is appropriate. The Clinical Data
Interchange Standards Consortium (CDISC) \citep{CDISC} develops
standards dealing with medical research data, including the submission
of trial results. Adhering to these standards is necessary for a
successful trial submission.

There are several data sets included with a submission that tend to be
useful for analysis. This paper focuses on the Analysis Data Model
(ADaM) data, which provides patient-level data, which has been validated
and used for data derivation and analysis. An ADaM data set is itself
composed of several data sets, including a Subject-Level Analysis Data
Set (ADSL) holding analysis and treatment information. Other
information, including baseline characteristics, demographic data, visit
information, etc., are held in the and Basic Data Structure (BDS)
formatted data sets. Finally, adverse events are held in the Analysis
Data Sets for Adverse Events (ADAE).

\hypertarget{challenges-to-analyzing-these-data-sets}{%
\subsection{Challenges to analyzing these data
sets}\label{challenges-to-analyzing-these-data-sets}}

ADaM data for a clinical trial is generally made available as a set of
SAS7BDAT \citep{sas7bdat} files. While neither the FDA nor the EMA
require this format for submission nor do they requires the use of SAS
\citep{sas} for analysis, there is a heavy bias toward the data format
and computing platform. This is partially because they are validated and
approved by governing bodies and because a large effort has gone into
their use in submissions. Packages like \pkg{sas7bdat} \citep{sasbdat}
and, more recently, \pkg{haven} \citep{haven} have gone a long way to
make these data sets easily accessible to R \citep{R} users working with
clinical trial data.

Despite the effort that has gone into defining a structure for the data
as well as the tools implemented to aid in their analysis, the data sets
themselves are not particularly easy to analyze for two reasons. First,
the standard is not ``tidy'' as defined by \citet{wickham2014}. In
particular, it is not required that each variable forms a column. In
fact, multiple variables may be stored in one column, with another
column acting as a key as to which variable's value is given. This case
is often seen in the ADSL data set where a single column may primary and
secondary endpoints. For data sets like these, the value variable is
held in the Analysis Value (AVAL) if the corresponding variable is
numeric, Analysis Value Character (AVALC) if the variable is a string,
the Parameter Character Description (PARAMCD) column giving a shorted
variable name, and the Parameter column providing a text description of
the variable. As an example, consider the \texttt{adakiep.xpt} data set,
which is provided as an example on the CDISC website and whose data is
included in the supplementary material.

\begin{Schunk}
\begin{Sinput}
library(readr)

adakiep <- read_csv("adakiep.csv") 
adakiep
\end{Sinput}
\begin{Soutput}
#> # A tibble: 24 x 8
#>    USUBJID     PARAM                PARAMCD AVALC   ADY ADT        SRCDOM SRCSEQ
#>    <chr>       <chr>                <chr>   <chr> <dbl> <date>     <chr>   <dbl>
#>  1 XYZ-001-001 Death                DEATH   Y        85 2013-11-02 DS          1
#>  2 XYZ-001-001 Dialysis             DIALYS~ Y        80 2013-10-29 PR          2
#>  3 XYZ-001-001 eGFR 25 Percent Dec~ EGFRDEC N        85 2013-11-02 <NA>       NA
#>  4 XYZ-001-001 Composite AKI Endpo~ AKIEP   Y        80 2013-10-29 <NA>       NA
#>  5 XYZ-001-002 Death                DEATH   Y        82 2015-03-20 DS          1
#>  6 XYZ-001-002 Dialysis             DIALYS~ Y        73 2015-03-11 PR          2
#>  7 XYZ-001-002 eGFR 25 Percent Dec~ EGFRDEC N        82 2015-03-20 <NA>       NA
#>  8 XYZ-001-002 Composite AKI Endpo~ AKIEP   Y        73 2015-03-11 <NA>       NA
#>  9 XYZ-001-003 Death                DEATH   N        94 2010-10-12 DS          1
#> 10 XYZ-001-003 Dialysis             DIALYS~ Y        64 2010-09-12 PR          2
#> # ... with 14 more rows
\end{Soutput}
\end{Schunk}

The data set includes minimal information about the trial. However, we
can infer that it is from a study focusing on kidney disease. There are
four distinct endpoints, death, whether dialysis was needed, whether a
25\% decrease in estimated glomerular filtration rate, indicating a
decrease in kidney function. For analysis, these data will need to be
re-arranged so that each endpoint has it's own column along with another
column per endpoint indicating the trial day where the measurement was
taken (from the ADY column).

The task of transforming these types of data into into appropriate
analysis is complicated by the fact that there may be other files with
relevant information with similar layout or layouts slightly more
complicated if they include longitudinal information, for example. The
rest of this paper focuses on shaping these types of data so that they
can be quickly understood; they are amenable to many different types of
analyses at the individual patient level; and they can be reformatted
for an even larger class of analyses through a minimal set of verbs,
including cohorting, which is introduced in this paper and is
implemented in the \pkg{forceps} package \citep{forceps}. The package is
currently in development and has not been released to CRAN. However, it
has been tagged for prerelease on Github and can be installed with the
following code.

\begin{Schunk}
\begin{Sinput}
devtools::install_github("kaneplusplus/forceps@v0.0.5")
\end{Sinput}
\end{Schunk}

\noindent The next section specifies the target data shape, which can be
thought of as a restriction on the tidy format. The following section
specifies the steps needed to prepare clinical trial data so that it
conforms to this restriction and includes an anonymized trial example.
The final section provides a roadmap if near-term development as well as
directions for enhancements and integration of the larger R ecosystem.

\hypertarget{a-tidy-representation-for-a-consolidated-analysis-data-set}{%
\section{A tidy representation for a consolidated analysis data
set}\label{a-tidy-representation-for-a-consolidated-analysis-data-set}}

Clinical trial data is collected to be used in an analysis that
determines whether or not a treatment for a disease provides a benefit
when compared to those receiving either a placebo or the standard of
care. ``Benefit'' is quantified by one or more endpoints, defined before
the trial starts (in the \emph{design}), which are compared across arms
(treatment and placebo) at the conclusion of the trial using a
statistical test. These data provide a wealth of information, and their
usefulness extends well beyond the scope of the trial. For example, they
can also be used to understand prognostic characteristics of the
disease-population; they can be used to create a ``historical
control'\,' for another trial; they can be used to identify patients'
characteristics associated with better outcomes, etc.

As shown in the previous section, while ADaM-formatted data is
structured, the structure does not lend itself to analysis without first
performing some data transformations. We propose that the result of
these transformations is a single data set with the following
characteristics.

\begin{enumerate}
\def\labelenumi{\arabic{enumi}.}
\tightlist
\item
  Each row corresponds to a single patient.
\item
  A variable with one value per patient should be included as a column
  variable.
\item
  Longitudinal, time series, or repeated measures data should be stored
  as an embedded \texttt{data.frame} per subject.
\end{enumerate}

\noindent Data conforming to these characteristics provide several
advantages to ADaM data sets. First, they are oriented towards trial
analysis. Essentially, trials compare response rates between treatment
and control arms. Having those values coded as their own variables in a
single data set minimizes the complexity and effort that would otherwise
go into extracting data from multiple files, cleaning them and joining
them. Second, it minimizes the reshaping effort for other types of
analyses. For example, response rates are often analyzed by the site at
which patient measurements were taken in order to check for certain
types of enrollment heterogeneity. The described patient-centric format
can be transformed into a site-centric format by nesting or grouping on
a site variable followed by the extraction of site-specific features and
analyses, which can then be compared across sites. Transforming between
these formats requires a single operation. Likewise, the patient-centric
format can be transformed to a patient-longitudinal format by unnesting
on the embedded variable holding the relevant longitudinal information.
Third and finally, creating a single patient-centric data set minimizes
the chance of inconsistent analyses. Primary and secondary analyses
often use similar variables and may require similar preprocessing. If
these preprocessing steps are performed separately for parallel
analyses, then the probability that at least one of them contains an
error in these steps is greater than when a validated patient-centric
data set is created. It also makes it easier to provide provenance for
an analysis if they are dependent on the same preprocessed data.

\hypertarget{processing-adam-data-to-reach-the-tidy-representation}{%
\section{Processing ADaM data to reach the tidy
representation}\label{processing-adam-data-to-reach-the-tidy-representation}}

This section provides an example of how to use the functionality
provided in the \pkg{forceps} package, in the order that the operations
take place. The data set is provided with the package, and the variable
names are taken from several example lung cancer studies. The data set
has been significantly reduced in size, and some values and variable
names have been preprocessed. This allows the example to remain easy to
follow. It also allows us to illustrate the formation of a
patient-centric data set in a single pass. In practice, this is often an
iterative process, requiring several revisions as bugs are found, and
hypotheses change.

The data sets used are as follows, and the task will be to create a
patient-centered data set as described above.

\begin{enumerate}
\def\labelenumi{\arabic{enumi}.}
\tightlist
\item
  \texttt{lc\_adverse\_events} - adverse events longitudinal data.
\item
  \texttt{lc\_biomarkers} - patient biomarkers.
\item
  \texttt{lc\_demography} - patient demographic information.
\item
  \texttt{lc\_adsl} - response data.
\end{enumerate}

\hypertarget{creating-the-data-dictionary}{%
\subsection{Creating the data
dictionary}\label{creating-the-data-dictionary}}

SAS ADaM formatted data sets generally include extra information about
variables, including a short description of each of the variables and
possibly formatting information. The \pkg{haven} package keeps this as
attributes of each of the columns of a \texttt{tibble} that is read from
these files. The \pkg{forceps} package is capable of extracting this
meta-information to create a \texttt{tibble} that can be used as a data
dictionary using the \texttt{consolidated\_describe\_data()} function,
shown below. In practice, we have found it helpful as a starting for a
fuller description of the data and often add columns to further
categorize individual variables for analyses.

\begin{Schunk}
\begin{Sinput}
library(forceps)

data(lc_adverse_events)
data(lc_biomarkers)
data(lc_demography)
data(lc_adsl)

consolidated_describe_data(lc_adverse_events,
                           lc_biomarkers,
                           lc_demography,
                           lc_adsl)
\end{Sinput}
\begin{Soutput}
#> # A tibble: 27 x 5
#>    var_name      type     label                      format_sas data_source     
#>    <chr>         <chr>    <chr>                      <chr>      <chr>           
#>  1 usubjid       double   Randomization Code         <NA>       lc_adverse_even~
#>  2 ae            charact~ AE Preferred Term          character  lc_adverse_even~
#>  3 ae_type       charact~ System Organ Class 1       character  lc_adverse_even~
#>  4 grade         integer  Adverse Event Grade?       numeric    lc_adverse_even~
#>  5 ae_day        double   Days From First Dose (num~ <NA>       lc_adverse_even~
#>  6 ae_duration   double   Adverse Event Duration     numeric    lc_adverse_even~
#>  7 ae_treat      logical  Was the Adverse Event Tre~ logical    lc_adverse_even~
#>  8 ae_count      integer  Total Patient Adverse Eve~ integer    lc_adverse_even~
#>  9 usubjid       double   Randomization Code         <NA>       lc_biomarkers   
#> 10 egfr_mutation charact~ EGFR Mutation +ve/-ve Res~ character  lc_biomarkers   
#> # ... with 17 more rows
\end{Soutput}
\end{Schunk}

\hypertarget{cohorting}{%
\subsection{Cohorting}\label{cohorting}}

The data dictionary (or data description) provides a summary of the
variable types and information held by variables in each of the data
sets. Some data sets will include repeated, longitudinal, or time series
information about individual patients, like
\texttt{lc\_adverese\_events} in our example. Consolidating data sets
like these into a single, patient-centric data set generally involves
three distinct operations. The first can be thought of as
\texttt{pivot\_wider()} operations that take columns composed of
multiple variables and spread them across new columns in the data set.
The second takes the data set and \texttt{nest()}'s the data so that the
the resulting data set contains time-varying data embedded in a
\texttt{data.frame} variable and those variables that are repeated
appear once per patient in the new variables. This verb, which is
referred to as \texttt{cohort()} in the package, takes the variable to
cohort on (the \texttt{usubjid} in the example below), checks for values
that are repeated by the subject identifier (\texttt{ae\_count} in the
example below) and those that are not, and handles the nesting
appropriately. A final operation may be applied to the patient-level
embedded \texttt{data.frame} objects to extract other features that will
be used in subsequent analyses.

\begin{Schunk}
\begin{Sinput}
library(dplyr)

data(lc_adverse_events)

lc_adverse_events %>% head()
\end{Sinput}
\begin{Soutput}
#> # A tibble: 6 x 8
#>   usubjid ae        ae_type           grade ae_day ae_duration ae_treat ae_count
#>     <dbl> <chr>     <chr>             <int>  <dbl>       <dbl> <lgl>       <int>
#> 1    1003 BURNING ~ NERVOUS SYSTEM D~     1     27           4 FALSE          15
#> 2    1003 CONSTIPA~ GASTROINTESTINAL~     2      4           4 TRUE           15
#> 3    1003 DEPRESSI~ PSYCHIATRIC DISO~     2     66          NA FALSE          15
#> 4    1003 BACK PAIN MUSCULOSKELETAL ~     2     27          NA TRUE           15
#> 5    1003 DYSURIA   RENAL AND URINAR~     2      1           3 TRUE           15
#> 6    1003 SKIN EXF~ SKIN AND SUBCUTA~     1      5          26 FALSE          15
\end{Soutput}
\begin{Sinput}
lc_adverse_events <- lc_adverse_events %>%
  cohort(on = "usubjid", name = "ae_long")

lc_adverse_events %>% head()
\end{Sinput}
\begin{Soutput}
#> # A tibble: 6 x 3
#>   usubjid ae_count ae_long          
#>     <dbl>    <int> <list>           
#> 1    1003       15 <tibble [15 x 6]>
#> 2    1005       19 <tibble [19 x 6]>
#> 3    1006       11 <tibble [11 x 6]>
#> 4    1009       12 <tibble [12 x 6]>
#> 5    1014        5 <tibble [5 x 6]> 
#> 6    1018       10 <tibble [10 x 6]>
\end{Soutput}
\end{Schunk}

\hypertarget{identifying-conflicts-and-redundancies}{%
\subsection{Identifying conflicts and
redundancies}\label{identifying-conflicts-and-redundancies}}

After cohorting, each of the data sets is in the specified format, and
we are almost ready to combine them. It is important to first check to
see if there are variables that are repeated across the individual data
sets and detect conflicts. While ADaM data sets \emph{should} be free of
conflicts and redundancies, we have observed multiple cases where this
is not true. In order to identify these issues, the
\texttt{duplicate\_vars()} function is provided. The function checks the
column names of each of the data sets with those of other column names.
The object returned is a named list where the name corresponds to the
variable that is repeated. Each list element returns a \texttt{tibble},
joined by the \texttt{on} parameter with columns corresponding to the
\texttt{on} variable, the duplicated variable, the data sets where the
duplicated variable appears. The example below shows that the
\texttt{chemo\_stop} variable appears in the \texttt{demography} and
\texttt{adsl} data sets. Furthermore, we can see that the values in each
of the data sets are different by looking at the correspondence between
the \texttt{demography} and \texttt{adsl} columns. To fix this and move
on, we will remove the variable from the \texttt{demography} data set.

\begin{Schunk}
\begin{Sinput}
data(lc_adsl)
data(lc_biomarkers)
data(lc_demography)
data_list <- list(demography = lc_demography, 
                  biomarkers = lc_biomarkers, 
                  adverse_events = lc_adverse_events, 
                  adsl = lc_adsl)
duplicated_vars(data_list, on = "usubjid")
\end{Sinput}
\begin{Soutput}
#> $chemo_stop
#> # A tibble: 558 x 4
#>    usubjid var        demography            adsl                 
#>      <dbl> <chr>      <chr>                 <chr>                
#>  1    1003 chemo_stop patient discontinued  adverse events       
#>  2    1005 chemo_stop treatment ineffective adverse events       
#>  3    1006 chemo_stop <NA>                  treatment ineffective
#>  4    1009 chemo_stop treatment ineffective <NA>                 
#>  5    1014 chemo_stop <NA>                  adverse events       
#>  6    1018 chemo_stop treatment ineffective treatment ineffective
#>  7    1023 chemo_stop <NA>                  adverse events       
#>  8    1025 chemo_stop adverse events        adverse events       
#>  9    1030 chemo_stop adverse events        adverse events       
#> 10    1033 chemo_stop adverse events        treatment ineffective
#> # ... with 548 more rows
\end{Soutput}
\begin{Sinput}
data_list$demography <- data_list$demography %>% 
  select(-chemo_stop)
\end{Sinput}
\end{Schunk}

\hypertarget{consolidating}{%
\subsection{Consolidating}\label{consolidating}}

The last step is to consolidate the data sets into a single one. This is
accomplished by reducing the \texttt{data\_list} using full joins, along
with some extra checking. The \texttt{consolidate()} function wraps this
functionality. The result, conforming to the provided format, which can
easily be used in the exploration and analysis stage.

\begin{Schunk}
\begin{Sinput}
consolidate(data_list, on = "usubjid")
\end{Sinput}
\begin{Soutput}
#> # A tibble: 558 x 18
#>    usubjid site_id sex   refractory   age egfr_mutation smoking ecog  prior_resp
#>      <dbl>   <int> <chr> <lgl>      <dbl> <chr>         <chr>   <chr> <chr>     
#>  1    1003       1 male  FALSE         51 negative      former~ ambu~ complete ~
#>  2    1005       4 fema~ TRUE          44 negative      former~ ambu~ partial r~
#>  3    1006       2 male  TRUE          22 negative      former~ ambu~ complete ~
#>  4    1009       8 male  FALSE         44 <NA>          unknown ambu~ complete ~
#>  5    1014       6 male  TRUE          76 <NA>          former~ ambu~ partial r~
#>  6    1018      10 fema~ TRUE          35 positive      former~ ambu~ complete ~
#>  7    1023       6 fema~ TRUE          73 <NA>          former~ ambu~ complete ~
#>  8    1025       7 male  FALSE         71 <NA>          never ~ ambu~ partial r~
#>  9    1030       5 fema~ TRUE          20 <NA>          unknown ambu~ partial r~
#> 10    1033       6 fema~ TRUE          55 <NA>          unknown ambu~ stable di~
#> # ... with 548 more rows, and 9 more variables: ae_count <int>, ae_long <list>,
#> #   best_response <chr>, pfs_days <dbl>, pfs_censor <dbl>, os_days <dbl>,
#> #   os_censor <dbl>, chemo_stop <chr>, arm <chr>
\end{Soutput}
\end{Schunk}

\hypertarget{direction-an-integrated-approach-to-processing-clinical-data}{%
\section{Direction: An integrated approach to processing clinical
data}\label{direction-an-integrated-approach-to-processing-clinical-data}}

As stated before, the goal of this paper is to help define a path toward
a grammar for processing clinical trials by a) defining a format in
which we would like to represent data from standardized clinical trial
data b) describing a standard set of operations to transform clinical
trial data into this format, and c) to identify a set of verbs and other
functionality to facilitate data processing of this kind and encourage
reproducibility of these steps. Admittedly, this only serves to mitigate
the process of preparing these types of data for exploration and
analysis. Clinical trial data generally contains many more variables
than what was presented, and each of these data sets comes with its own
set of ``quirks'' and other challenges. However, it does serve to make
the data preparation better defined and propose a path toward
standardization of both the processed data set format as well as the
operations to achieve that goal.

Along with further development towards those ends, there is a plethora
of development that can be done to provide an integrated data processing
experience. For example, the define.xml file, which appears alongside
ADaM data sets, gives better descriptions of the variables as well as
the variable values. Tools to integrate these data into the construction
of the data dictionary would go a long way towards orienting researchers
with the data contained and help them more quickly formulate analyses.
Packages like \pkg{lumberjack} \citep{lumberjack} could enhance and
augment data preprocessing steps by keeping better track of when data
are being removed and how they are being manipulated. The artifacts
accumulated could then be used by packages such as \pkg{ggconsort}
\citep{ggconsort} to provide consort diagrams of how patients progress
through the trial and how data progresses through preprocessing. In the
longer term, these advancements can provide better data provenance, more
reproducible processing, quicker debugging of problems in the processing
stage, and give rise to more effective and convenient tools for
summarizing trial data.

\bibliography{forceps.bib}

\address{%
Michael J. Kane\\
Yale University\\%
60 College Street\\ New Haven, CT 06510, USA\\
%
%
%
\\\href{mailto:michael.kane@yale.edu}{\nolinkurl{michael.kane@yale.edu}}
}
