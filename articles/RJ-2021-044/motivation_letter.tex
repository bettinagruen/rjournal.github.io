\documentclass[11pt, a4paper,draft]{article}

\usepackage[utf8]{inputenc}
\usepackage{amsmath,amssymb,amsfonts,amsthm}
\usepackage{bbm}
\usepackage{enumerate}
\usepackage{graphicx}
\usepackage{geometry}
\usepackage{color}
\usepackage{url}

\begin{document} 

Dear Editors,

\vskip 12pt

I am writing as the corresponding author of the article for the add-on package under submission to {\it R Journal} 
entitled
"OneStep - Le~Cam's one-step estimation procedure" by A. Brouste, C. Dutang and D. Noutsa Mieniedou.

In the statistical experiments generated by an i.i.d. observation sample, the sequence of maximum likelihood estimators (MLE) is known to be asymptotically efficient under very general assumptions and consequently presents the fastest convergence rate and the lowest possible asymptotic variance.

Although the sequence of MLE is asymptotically efficient, it is generally not expressed in a closed form and requires time consuming numerical computations. On the other hand, the other generic estimation procedures which can sometimes be computed faster, as the method of moments, do not generally reach the optimal asymptotic variance.

But, as soon as the Fisher information matrix is sufficiently regular with respect to the parameter to be estimated, Le Cam's one-step estimation procedure \cite{LeCam} and its recent improvement \cite{Uchida} can be achieved. They are based on an initial sequence
of guess estimators and a single Newton step or a Fisher scoring step on the loglikelihood function. 

The sequence of Le Cam's one-step estimators presents certain advantages over the sequence of MLE and over the initial sequence of estimators (method of moments, etc.) in terms of computational cost and asymptotic variance. 
It is much less computationally expensive than the MLE while it has the same rate and the same asymptotic variance. 
Since there is no full numerical optimization (but only one computation of the Newton step or the Fisher scoring step), the procedure 
is faster and appropriate for very large datasets.
On the other hand, it is asymptotically  optimal in terms of asymptotic variance which is generally not 
the case for the initial sequence of guess estimators. 

In R,  the package {\bf fitdistrplus} is commonly used to infer the parameters 
of univariate probability distributions. Other packages with similar purposes are  {\bf EstimationTools} and {\bf DistributionFitR} 
providing MLE for non-censored data.

The Le Cam procedure is not implemented in R despite its obvious aforementioned advantages. The OneStep package fills this gap with a panel of distributions for which initial sequence, score and Fisher information matrix have been computed or collected by our team.

Thank you in advance for your work.

\thispagestyle{empty}

\vskip 8 pt

\begin{flushright}
Sincerely yours,
\end{flushright}

\vskip 8 pt

\begin{flushright}
Alexandre Brouste 
\end{flushright}

{\footnotesize

\begin{thebibliography}{00}


\bibitem{LeCam} {Le~Cam, L.} (1956)  {\it On the asymptotic theory of estimation and testing hypotheses.}In Proceedings of the Third Berkeley Symposium on Mathematical Statistics and Probability, Volume 1: Contributions to the Theory of Statistics, pages 129–156, University of California Press.  

\bibitem{Uchida}{Kamatani, K. and Uchida, M.} (2015) {\it Hybrid multi-step estimators for stochastic differential equations based
on sampled data.} Stat. Inference Stoch. Process, 18(2):177–204.

%\bibitem{Coeurjolly} {C{\oe}urjolly, J.-F. and Istas, J.} (2001) {\it Cramer-Rao bounds for fractional Brownian motions}, Statistics \& Probability Letters, 53, 435--447.
 

  
\end{thebibliography} }

\end{document}