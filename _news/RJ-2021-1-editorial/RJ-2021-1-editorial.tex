% !TeX root = RJwrapper.tex
\title{Editorial}
\author{by Dianne Cook}

\maketitle


On behalf of the editorial board, I am pleased to present Volume 13
Issue 1 of the R Journal.

First, some news about the journal board. Welcome to Gavin Simpson, who
joins as a new Executive Editor! In addition, welcome to our new
Associate Editors Nicholas Tierney, Isabella Gollini, Rasmus
B\r{a}\r{a}th, Mark van der Loo, Elizabeth Sweeney, Louis Aslett and
Katarina Domijan. With the large volume of submissions, the Associate
Editors now play a vital role in processing articles.

There are some new developments in the journal operations under way. We
are working on a new package \pkg{rjtools} which will operate a little
like the \CRANpkg{devtools} package and help you to create a new article
from a template, and check that it conforms to the style and
requirements of the R Journal.

We are also working on supporting articles written in RMarkdown, which
will be rendered in html through a modified \CRANpkg{distill} web site.
The exciting feature is that interactive graphics could be included
directly in the article. You can see how this current issue would look
in the new style at \url{https://rjournal.r-project.org/dev}.
Particularly, look at articles Conversations in Time by Wang and Cook as
an example that has two examples of how interactive graphics might be
included. Other articles rendered in html are ``Finding Optimal
Normalizing Transformations'' by Peterson, ``Automating Reproducible,
Collaborative Clinical Trial Document Generation'' by Kane, Jiang and
Urbanek, and ``Towards a Grammar for Processing Clinical Trial Data'' by
Kane. All remaining articles in the new site style are the current pdf
style.

To experiment with creating a new article, or to check that your
article, conforms with the R Journal author guidelines, go to
\url{https://rjournal.github.io/rjtools/}. Note that it is still ok to
use the \CRANpkg{rticles} package R Journal Rmarkdown template to create
your article. This will generate the files that are compiled to pdf
using latex, but it is an easy translation for us to convert them into
the new style.

The operational support and the experiments have been supported with
generous funding from the R Consortium
(\url{https://www.r-consortium.org}).

Behind the scenes, several people are assisting with the journal
operations and the new developments. Mitchell O'Hara-Wild has worked on
infrastructure, the new article submission system, a new issue build
system and now the new article delivery system providing html format. H.
Sherry Zhang has taken over from Stephanie Kobakian, in developing the
\pkg{rjtools} package including check functions for new articles to help
authors get the style constraints correct. In addition, articles in this
issue have been painstakingly copy edited by Dewi Amaliah.

\hypertarget{in-this-issue}{%
\section{In this issue}\label{in-this-issue}}

News from the R Core, CRAN, Bioconductor, the R Foundation, and the
foRwards Taskforce are included in this issue along with a summary of
activities at the R Medicine and Why R? 2021 conferences.

This issue features 37 contributed research articles covering these
topics:

\begin{itemize}
\tightlist
\item
  Multivariate analysis

  \begin{itemize}
  \tightlist
  \item
    \CRANpkg{SeedCCA}: An integrated R-package for Canonical Correlation
    Analysis and Partial Least Squares
  \item
    Unidimensional and Multidimensional Methods for Recurrence
    Quantification Analysis with \CRANpkg{crqa}
  \item
    \CRANpkg{clustcurv}: An R Package for Determining Groups in Multiple
    Curves
  \item
    \CRANpkg{gofCopula}: Goodness-of-Fit Tests for Copulae
  \item
    \CRANpkg{ROCnReg}: An R Package for Receiver Operating
    Characteristic Curve Inference With and Without Covariates
  \end{itemize}
\item
  Non-parametric methods

  \begin{itemize}
  \tightlist
  \item
    \CRANpkg{npcure}: An R Package for Nonparametric Inference in
    Mixture Cure Models
  \item
    ROBustness In Network (\CRANpkg{robin}): an R package for Comparison
    and Validation of Communities
  \item
    \CRANpkg{krippendorffsalpha}: An R Package for Measuring Agreement
    Using Krippendorff's Alpha Coefficient
  \end{itemize}
\item
  Temporal and longitudinal methods

  \begin{itemize}
  \tightlist
  \item
    \CRANpkg{JMcmprsk}: An R Package for Joint Modelling of Longitudinal
    and Survival Data with Competing Risks
  \item
    Linear Regression with Stationary Errors: the R Package
    \CRANpkg{slm}
  \item
    \CRANpkg{penPHcure}: Variable Selection in Proportional Hazards Cure
    Model with Time-Varying Covariates
  \item
    \CRANpkg{pdynmc}: A Package for Estimating Linear Dynamic Panel Data
    Models Based on Nonlinear Moment Conditions
  \item
    \CRANpkg{DChaos}: An R Package for Chaotic Time Series Analysis
  \item
    \CRANpkg{IndexNumber}: An R Package for Measuring the Evolution of
    Magnitudes
  \item
    \CRANpkg{garchx}: Flexible and Robust GARCH-X Modelling
  \item
    Working with CRSP/COMPUSTAT in R: Reproducible Empirical Asset
    Pricing
  \item
    Analysing Dependence Between Point Processes in Time Using
    \CRANpkg{IndTestPP}
  \item
    Conversations in Time: Interactive Visualisation to Explore
    Structured Temporal Data
  \end{itemize}
\item
  Computing infrastructure

  \begin{itemize}
  \tightlist
  \item
    A Method for Deriving Information from Running R Code
  \item
    Wide-to-tall Data Reshaping Using Regular Expressions and the
    \CRANpkg{nc} Package
  \item
    The \CRANpkg{bdpar} Package: Big Data Pipelining Architecture for R
  \item
    Benchmarking R packages for calculation of Persistent Homology
  \item
    \CRANpkg{distr6}: R6 Object-Oriented Probability Distributions
    Interface in R
  \item
    Automating Reproducible, Collaborative Clinical Trial Document
    Generation
  \item
    Reproducible Summary Tables with the \CRANpkg{gtsummary} Package
  \item
    Towards a Grammar for Processing Clinical Trial Data
  \end{itemize}
\item
  Simulation and optimisation

  \begin{itemize}
  \tightlist
  \item
    Finding Optimal Normalizing Transformations via
    \CRANpkg{bestNormalize}
  \item
    Package \CRANpkg{wsbackfit} for Smooth Backfitting Estimation of
    Generalized Structured Models
  \item
    \CRANpkg{RLumCarlo}: Simulating Cold Light using Monte Carlo Methods
  \item
    \CRANpkg{OneStep}: Le Cam's one-step estimation procedure
  \item
    The \CRANpkg{HBV.IANIGLA} Hydrological Model
  \item
    Regularized Transformation Models: The \CRANpkg{tramnet} Package
  \end{itemize}
\item
  Other topics

  \begin{itemize}
  \tightlist
  \item
    \CRANpkg{exPrior}: An R Package for the Formulation of Ex-Situ
    Priors
  \item
    \CRANpkg{BayesSPsurv}: An R Package to Estimate Bayesian (Spatial)
    Split-Population Survival Models
  \item
    Statistical Quality Control with the \CRANpkg{qcr} Package
  \item
    The R Package \CRANpkg{smicd}: Statistical Methods for
    Interval-Censored Data
  \item
    \CRANpkg{stratamatch}: Prognostic Score Stratification Using a Pilot
    Design
  \end{itemize}
\end{itemize}

Happy reading, and code testing!


\address{%
Dianne Cook\\
Monash University\\%
\\
%
\url{https://journal.r-project.org}\\%
%
\href{mailto:r-journal@r-project.org}{\nolinkurl{r-journal@r-project.org}}%
}
