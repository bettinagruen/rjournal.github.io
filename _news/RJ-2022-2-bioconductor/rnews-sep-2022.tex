
\title{News from the Bioconductor Project}
\author{by Bioconductor Core Team}
\maketitle

\href{https://bioconductor.org}{Bioconductor} provides
tools for the analysis and comprehension of high-throughput genomic
data.  The project has entered its twentieth year, with funding
for core development and infrastructure maintenance secured
through 2025 (NIH NHGRI 2U24HG004059).  Additional support is provided
by NIH NCI, Chan-Zuckerberg Initiative, National Science Foundation,
Microsoft, and Amazon.  In this news report, we give some
details about the software and data resource collection,
infrastructure for building, checking, and distributing resources,
core team activities, and some new initiatives.
 
\textit{Software ecosystem}

Bioconductor 3.15 was released on 27 April, 2022. It is
compatible with R 4.2.0 and consists of 2140 software packages, 410
experiment data packages, 990 up-to-date annotation packages, 29
workflows, and 3 books. \href{https://bioconductor.org/books/release/}{Books} are
built regularly from source and therefore fully
reproducible; an example is the
community-developed \href{https://bioconductor.org/books/release/OSCA/}{Orchestrating
Single-Cell Analysis with Bioconductor}.
The Bioconductor
\href{https://bioconductor.org/news/bioc_3_15_release/}{3.15 release
  announcement} includes descriptions of 78 new software packages,
and updates to NEWS files for many additional packages.  

\textit{Infrastructure updates}

\begin{itemize}
\item Thanks to a generous allocation (BIR190004, "Engineering and disseminating a software and analysis ecosystem for genomic data science") provided through the National Science
Foundation ACCESS (formerly XSEDE) program, academic cloud resources including
GPUs and highly accessible object storage systems are being integrated
into project operations.
\item Transition of primary funding administration from Roswell Park Comprehensive
Cancer Center to Dana-Farber Cancer Institute has led to a number of changes to
platforms in use for the checking and production of binary package images.
\begin{itemize}
\item Linux builds occur at Dana-Farber Cancer Institute.
\item Windows builds occur in machinery provided by Microsoft Genomics
in the Azure cloud environment.
\item MacOS builds occur at Dana-Farber Cancer Institute.  Work on the
support of ARM Mac systems occurs at MacStadium.
\item Details on the configurations of builders (e.g.,
\href{https://bioconductor.org/checkResults/3.16/bioc-LATEST/nebbiolo2-NodeInfo.html}{the Linux builder}
for the devel branch) are available at the
\href{https://bioconductor.org/checkResults/}{Build reports} link at bioconductor.org.
\end{itemize}
\item An interactive app for surveying adverse conditions arising
for package install, build, and check processes has been introduced
for \href{https://vjcitn.shinyapps.io/biocPkgState315/}{release} and
\href{https://vjcitn.shinyapps.io/biocPkgState316/}{devel} branches.
\item Cloud-based workshop delivery systems have been an integral part
of Bioconductor conferences and teaching activities.
\begin{itemize}
\item Workshops from Bioconductor 2022 are continuously available for
inspection and hands-on exercises at \url{http://app.orchestra.cancerdatasci.org}, thanks
to cloud computing support provided by Dr. Sean Davis of University of Colorado.
\item \url{http://workshop.bioconductor.org} is a Galaxy-based workshop collection
deployed on Jetstream2 in NSF ACCESS.
\end{itemize}
\end{itemize}

\textit{Core team updates}
\begin{itemize}
\item After six years of highly effective work in the core, Nitesh
Turaga has left for a position in industry.  We will miss him!
\item New core developers Jen Wokaty and Alexandru Mahmoud have
joined.  Jen is a member of the Waldron Lab at CUNY.  Alex works
at Channing Division of Network Medicine.
\item Jen and Alex are joined by long-term core members Lori Kern of Roswell Park
Comprehensive Cancer Center, Marcel Ramos of CUNY and Roswell,
and Herv\'e Pages of Fred Hutchinson Cancer Research Center.
\end{itemize}

\textit{New initiatives}

\begin{itemize}
\item Thanks to efforts of members of the Technical and Community Advisory
Boards and community members, a collection of working groups has been defined
to achieve new project aims.  An \href{http://workinggroups.bioconductor.org/currently-active-working-groups-committees.html}{overview} of currently active working
groups is available, along with \href{http://workinggroups.bioconductor.org/working-group-and-committee-guidelines.html}{guidelines for proposing new working groups}.
\item The objectives of the bioconductor-teaching working group are stated at
the associated \href{https://github.com/bioconductor/bioconductor-teaching}{repository}:
\begin{quotation}
The Bioconductor teaching committee is a collaborative effort to consolidate Bioconductor-focused training material and establish a community of Bioconductor trainers. We define a curriculum and implement online lessons for beginner and more advanced R users who want to learn to analyse their data with Bioconductor packages.
\end{quotation}
\item A \href{https://bioconductor.org/developers/new-developer-program/}{mentoring program} for new
developers has taken flight.
\item Thanks to an Essential Open Source Software grant from the Chan-Zuckerberg Initiative, 
we have partnered with the Dana-Farber Cancer Institute \href{https://www.dfhcc.harvard.edu/research/cancer-disparities/students/yes-for-cure/}{YES for CURE} (Young Empowered Scientists for Continued Research Engagement)
program to offer instruction in cancer data science to interested undergraduates.  A
\href{https://vjcitn.github.io/YESCDS/}{pkgdown site} includes current curricular materials.
\item With the NSF-based academic cloud resources previously mentioned, we have begun gestation
of G-DADS, a program for Genomic Data and Analysis Development Services, with the objectives
of providing publicly accessible storage and compute on exemplars of the latest
high-volume experimental modalities, and of promoting GPUs to first-class citizenship
in our build and check systems.
\end{itemize}





\textit{Using Bioconductor}

Start using
Bioconductor by installing the most recent version of R and evaluating
the commands
\begin{example}
  if (!requireNamespace("BiocManager", quietly = TRUE))
      install.packages("BiocManager")
  BiocManager::install()
\end{example}
Install additional packages and dependencies,
e.g., \BIOpkg{SingleCellExperiment}, with
\begin{example}
  BiocManager::install("SingleCellExperiment")
\end{example}
\href{https://bioconductor.org/help/docker/}{Docker}
images provides a very effective on-ramp for power users to rapidly
obtain access to standardized and scalable computing environments.
Key resources include:
%% 
\begin{itemize}
\item \href{https://bioconductor.org}{bioconductor.org} to install,
  learn, use, and develop Bioconductor packages.
\item A list of \href{https://bioconductor.org/packages}{available
  software}, linking to pages describing each package.
\item A question-and-answer style
  \href{https://support.bioconductor.org}{user support site} and
  developer-oriented
  \href{https://stat.ethz.ch/mailman/listinfo/bioc-devel}{mailing
    list}.
\item A community slack (\href{https://bioc-community.herokuapp.com/}{sign up})
   for extended technical discussion.
\item The
  \href{https://f1000research.com/channels/bioconductor}{F1000Research
    Bioconductor channel} for peer-reviewed Bioconductor work flows.
\item The \href{https://www.youtube.com/user/bioconductor}{Bioconductor YouTube} 
     channel includes recordings of keynote and talks from recent 
     conferences including Bioc2022, EuroBioC2022, and BiocAsia2021, in addition to 
     video recordings of training courses. 
\item Our \href{https://github.com/Bioconductor/Contributions}{package
  submission} repository for open technical review of new packages.
\end{itemize}

Recent Bioconductor conferences include
\href{https://bioc2022.bioconductor.org}{BioC 2022} (July 27-29),
and 
\href{https://eurobioc2022.bioconductor.org/}{European Bioconductor Meeting}
(September 14-16). Each had invited and contributed talks, as well as
workshops and other sessions to enable community
participation. Slides, videos, and workshop material for each
conference are, or will soon be, available on each conference web site
as well as from
the \href{http://bioconductor.org/help/course-materials/}{Courses and
Conferences} section of the Bioconductor web
site.

The Bioconductor project continues to mature as a
community. The \href{https://bioconductor.org/about/technical-advisory-board/}{Technical}
and \href{https://bioconductor.org/about/community-advisory-board/}{Community}
Advisory Boards provide guidance to ensure that the project addresses
leading-edge biological problems with advanced technical approaches,
and adopts practices (such as a
project-wide \href{https://bioconductor.org/about/code-of-conduct/}{Code
of Conduct}) that encourages all to participate. We look forward to
welcoming you!

\address{Bioconductor Core Team \\
    Channing Division of Network Medicine \\
    Mass General Brigham \\
    Harvard Medical School, Boston, MA \\ \\
    Department of Data Science \\
    Dana-Farber Cancer Institute \\
    Harvard Medical School, Boston, MA \\ \\
    Biostatistics and Bioinformatics \\
    Roswell Park Comprehensive Cancer Center, Buffalo, NY \\ \\
    Fred Hutchinson Cancer Research Center, Seatlle, WA \\ \\
    CUNY Graduate School of Public Health, New York, NY}

\email{maintainer@bioconductor.org}
