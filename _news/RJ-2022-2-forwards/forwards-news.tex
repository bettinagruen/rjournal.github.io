% !TeX root = RJwrapper.tex
\title{News from the Forwards Taskforce}
\author{by Heather Turner}

\maketitle


\href{https://forwards.github.io/}{Forwards} is an R Foundation
taskforce working to widen the participation of under-represented groups
in the R project and in related activities, such as the \emph{useR!}
conference. This report rounds up activities of the taskforce during the
first half of 2022.

\hypertarget{accessibility}{%
\subsection{Accessibility}\label{accessibility}}

As another step towards improving the accessibility of the R Journal, Di
Cook and Heather Turner are mentoring a Google Summer of Code student,
Abhishek Ulayil, on the project
\href{https://github.com/rstats-gsoc/gsoc2022/wiki/Converting-past-R-Journal-articles-to-HTML}{Converting
past R Journal articles to HTML}. This project has benefited from
regular input from Mitchell O'Hara Wild and Christophe Dervieux, authors
of the \pkg{rjtools} package that provides the new HTML template for R
Journal articles \citep{rjtools}.

s gwynn sturdevant and Jonathan Godfrey were part of an invited panel at
\emph{JSM 2022} on
\href{https://ww2.amstat.org/meetings/jsm/2022/onlineprogram/AbstractDetails.cfm?abstractid=319247}{Delivering
Data Differently} that explored alternatives to data visualisation.

\hypertarget{community-engagement}{%
\subsection{Community engagement}\label{community-engagement}}

The community team have taken a number of actions to support the R
community in Africa. A WhatsApp group has been set up for leaders of
African R User Groups, to facilitate collaboration. Kevin O'Brien has
been a Zoom host for several R User Groups, including the Botswana,
Eswatini and Bulawayo groups. Zane Dax worked with the Accra R User
Group on graphical design for advertising their meetups. Kevin O'Brien
and Sam Toet helped to organize the first
\href{https://satrdays.org/blog/2022/07/20/2022-francophone/}{Francophone
satRday}, which featured a line up of African speakers.

Another focus has been the relaunch of
\href{https://rainbowr.netlify.app/}{RainbowR} led by Ella Kaye and Zane
Dax. Following a well-attended online meetup, the website was rebuilt,
the Slack group opened to new members with a new code of conduct, and
the Twitter account has been in active use. The group plan to have
regular online meetups and to raise awareness of issues affecting the
LGBTQ+ community through sharing relevant data sets for exploration and
teaching.

Beyond this, taskforce members continue to engage with a range of
communities. Zane Dax contributed to an update of the
\href{https://mircommunity.com/about/}{Minorities in R (MiR)} website.
Yanina Bellini Saibene and Heather Turner joined the
\href{https://github.com/AsiaR-community/2022-inclusive_communities}{Building
inclusive communities} panel at the launch of the AsiaR community, to
share their experience from working with different communities.

\hypertarget{conferences}{%
\subsection{Conferences}\label{conferences}}

Yanina Bellini Saibene assisted the \emph{useR! 2022} team on behalf of
the R Foundation, to share expertise in the organization of virtual
conferences and help incorporate good practices that encourage diverse
participation. (The conference was originally planned to be hybrid, but
moved to be completely online.) Such practices included adding
representatives from different regions to the organizing team, securing
funding to caption talks, accepting elevator pitches and tutorials in
non-English languages, and adjusting the registration fees to the income
group of each participant's country of residence. Many of these
practices were based on the work of the organizing team of \emph{useR!
2021} - that included Yanina and other Forwards members - who summarized
their recommendations in the recent publication: ``Ten simple rules to
host an inclusive conference'' \citep{10sr}.

\hypertarget{r-contribution}{%
\subsection{R Contribution}\label{r-contribution}}

Saranjeet Kaur Bhogal and Heather Turner organized a series of
\href{https://contributor.r-project.org/events/collaboration-campfires}{Collaboration
Campfires} with a goal to demystify the R development process and
highlight ways that R programmers can contribute. The first two sessions
explored R's bug-tracking process and how R users can contribute to
reviewing bugs. The second two sessions explored R's process for
localization and how to contribute to a translation team. The sessions
attracted a diverse group of participants who engaged with the
interactive activities, providing a foundation for further engagement.

The R Contribution working Group (RCWG) organized a
\href{https://contributor.r-project.org/events/bug-bbq}{Bug BBQ} as a
satellite to \emph{useR! 2022}. Members of the RCWG prepared a number of
open bugs in advance, for participants to look at in one or more of
three organized online sessions. The event was supported by several R
Core members and attended by both novice and experienced contributors.
For experienced contributors the event provided a spur to work on open
bugs, leading to progress on several issues. Meanwhile novice
contributors contributed to the analysis of open bugs, under the
guidance of experienced contributors. As the first event of its kind,
the event showed promise as a way to engage the wider R community in
contribution.

Saranjeet Kaur Bhogal has been working on a new chapter for the
\href{https://contributor.r-project.org/rdevguide/}{R Development Guide}
on contributing translations, as part of the R project's
\href{https://github.com/rstats-gsod/gsod2022/wiki/GSOD-2022-Proposal}{Google
Summer of Docs project}, with substantial contribution from Michael
Chirico. Along with Ben Ubah, Michael is co-mentoring a Google Summer of
Code student, Meet Bhatnagar, to create a dashboard to
\href{https://github.com/rstats-gsoc/gsoc2022/wiki/Track-R-Translations-Status}{monitor
the status of translations in R}.

\hypertarget{changes-in-membership}{%
\subsection{Changes in Membership}\label{changes-in-membership}}

\hypertarget{new-members}{%
\subsubsection{New members}\label{new-members}}

We welcome the following member to the taskforce:

\begin{itemize}
\tightlist
\item
  Community team: Ella Kaye (co-leader)
\end{itemize}

\hypertarget{previous-members}{%
\subsubsection{Previous members}\label{previous-members}}

The following members have stepped down:

\begin{itemize}
\tightlist
\item
  Community team: Richard Ngamita (co-leader)
\item
  Accessibility team: Becca Wilson
\item
  Surveys team: Anna Vasylytsya (co-leader)
\end{itemize}

\noindent We thank them for their contribution to the taskforce.

\bibliography{RJreferences.bib}

\address{%
Heather Turner\\
University of Warwick\\%
UK\\
%
%
%
\href{mailto:Heather.Turner@R-Project.org}{\nolinkurl{Heather.Turner@R-Project.org}}%
}
