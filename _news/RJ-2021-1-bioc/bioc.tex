\title{News from the Bioconductor Project}
\author{by Bioconductor Core Team}
\maketitle

\href{https://bioconductor.org}{Bioconductor} provides
tools for the analysis and comprehension of high-throughput genomic
data. Bioconductor 3.13 was released on 20 May, 2021. It is
compatible with R 4.1.0 and consists of 2042 software packages, 406
experiment data packages, 965 up-to-date annotation packages, and 29
workflows. 

\href{https://bioconductor.org/books/release/}{Books} were introduced in
Bioconductor 3.12 and production continues in this release.
These are built regularly from source and therefore fully
reproducible; an example is the
community-developed \href{https://bioconductor.org/books/release/OSCA/}{Orchestrating
Single-Cell Analysis with Bioconductor}.

The Bioconductor
\href{https://bioconductor.org/news/bioc_3_13_release/}{3.13 release
  announcement} includes descriptions of 133 new software packages,
and updates to NEWS files for many additional packages.  Start using
Bioconductor by installing the most recent version of R and evaluating
the commands
\begin{example}
  if (!requireNamespace("BiocManager", quietly = TRUE))
      install.packages("BiocManager")
  BiocManager::install()
\end{example}
Install additional packages and dependencies,
e.g., \BIOpkg{SingleCellExperiment}, with
\begin{example}
  BiocManager::install("SingleCellExperiment")
\end{example}
\href{https://bioconductor.org/help/docker/}{Docker}
images provides a very effective on-ramp for power users to rapidly
obtain access to standardized and scalable computing environments.

Key learning resources include:
%% 
\begin{itemize}
\item \href{https://bioconductor.org}{bioconductor.org} to install,
  learn, use, and develop Bioconductor packages.
\item A list of \href{https://bioconductor.org/packages}{available
  software}, linking to pages describing each package.
\item A question-and-answer style
  \href{https://support.bioconductor.org}{user support site} and
  developer-oriented
  \href{https://stat.ethz.ch/mailman/listinfo/bioc-devel}{mailing
    list}.
\item A community slack (\href{https://bioc-community.herokuapp.com/}{sign up})
   for extended technical discussion.
\item The
  \href{https://f1000research.com/channels/bioconductor}{F1000Research
    Bioconductor channel} for peer-reviewed Bioconductor work flows.
\item The \href{https://www.youtube.com/user/bioconductor}{Bioconductor YouTube} 
     channel includes recordings of keynote and talks from recent 
     conferences including Bioc2020 and BiocAsia2020, in addition to 
     video recordings of training courses. 
\item Our \href{https://github.com/Bioconductor/Contributions}{package
  submission} repository for open technical review of new packages.
\end{itemize}

The \href{https://bioc2021.bioconductor.org/}{2021 Bioconductor conference} will be virtual,
August 4-6, 2021.

In conjunction with the \href{https://twitter.com/RBioinformatica}{Mexican Bioinformatics Network} and 
the \href{https://twitter.com/nnb_unam}{Nodo Nacional de Bioinformática CCG UNAM},
the Comunidad de Desarrolladores de Software en Bioinformática
have arranged two week-long
\href{https://comunidadbioinfo.github.io/post/cdsb-2021-workshops/#.YOgqyhNuelY}{online workshops}
addressing development of 
\href{https://comunidadbioinfo.github.io/cdsb2021_workflows/}{workflows with RStudio and shiny} and
\href{https://comunidadbioinfo.github.io/cdsb2021_scRNAseq/}{analysis of single-cell RNA-seq experiments},
August 9-13, 2021.

BiocAsia 2021 will be held November 1-4 2021 as a virtual event
The website and call for contributed talks are not open yet.
Keep an eye on \href{https://bioconductor.org/help/events/}{the events page}
for updates.  The \href{https://sites.google.com/view/biopackathon} Biopackathon
project has many points of contact with Bioconductor and recurs monthly.

The National Human Genome Research Institute's Analysis and Visualization
Laboratory (\href{https://anvilproject.org}{AnVIL}) is developing with contributions from Bioconductor
core team members.  A \href{https://docs.google.com/document/u/2/d/e/2PACX-1vSVGCaX-wnWyu1TUhhbsoVeTCJ6ODLG53OeMHKRbewGQOqOcMTnZQl7_jrR9kqOPQPlsFN1ecLT4lhd/pub}{series of recorded workshops} on the use of
Bioconductor to explore this cloud
computing system is available; additional workshops will be presented in the Fall of 2021.



The Bioconductor project continues to mature as a
community. The \href{https://bioconductor.org/about/technical-advisory-board/}{Technical}
and \href{https://bioconductor.org/about/community-advisory-board/}{Community}
Advisory Boards provide guidance to ensure that the project addresses
leading-edge biological problems with advanced technical approaches,
and adopts practices (such as a
project-wide \href{https://bioconductor.org/about/code-of-conduct/}{Code
of Conduct}) that encourages all to participate. We look forward to
welcoming you!

\address{Bioconductor Core Team \\
    Biostatistics and Bioinformatics \\
    Roswell Park Comprehensive Cancer Center, Buffalo, NY \\
    USA}
\email{maintainer@bioconductor.org}
