\documentclass[a4paper]{report}
\usepackage[utf8]{inputenc}
\usepackage[T1]{fontenc}
\usepackage{RJournal}
\usepackage[round]{natbib}
\bibliographystyle{abbrvnat}
\usepackage{amsmath,amssymb,array}
\usepackage{booktabs}

%% load any required packages here

\begin{document}

%% do not edit, for illustration only
\sectionhead{Contributed research article}
\volume{XX}
\volnumber{YY}
\year{20ZZ}
\month{AAAA}

%% replace RJtemplate with your article
\begin{article}

\title{Test case with a lot of text}
\author{by baconipsum.com}

\maketitle

\abstract{
  Flank commodo cupidatat, ut rump incididunt in dolore nostrud fugiat tempor fatback cillum pancetta. Brisket fugiat sint est dolor mollit turducken, beef ribs pancetta eiusmod rump deserunt ex. Brisket est pork chop pancetta ut strip steak eiusmod frankfurter swine t-bone beef shank irure sausage ribeye. Ut drumstick et pork jowl ut.
}

Bacon ipsum dolor sit amet nulla shoulder turducken excepteur ullamco, irure enim in sed nisi ad venison ea commodo. Beef ribs ribeye sirloin ullamco. Pork eu venison ham meatball cupidatat. In cupidatat spare ribs pastrami shank short ribs jowl short loin occaecat ullamco brisket.

\section{Examples}

The following example environment should match lines regardless of whether tabs or spaces are used:

\begin{example}
  a
  b 
\end{example}

Here's a reasonably long code chunk

\begin{example}
#' The length of a string (in characters).
#'
#' @inheritParams str_detect
#' @return numeric vector giving number of characters in each element of the
#'   character vector.  Missing string have missing length.
#' @keywords character
#' @seealso \code{\link\{nchar}\} which this function wraps
#' @export
#' @examples
#' str_length(letters)
#' str_length(c("i", "like", "programming", NA))
str_length <- function(string) {
  string <- check_string(string)

  nc <- nchar(string, allowNA = TRUE)
  is.na(nc) <- is.na(string)
  nc
}
\end{example}

\end{article}

\end{document}
